%
% Copyright � 2012 Peeter Joot.  All Rights Reserved.
% Licenced as described in the file LICENSE under the root directory of this GIT repository.
%
%
%
%\input{../peeter_prologue_print.tex}
%\input{../peeter_prologue_widescreen.tex}
\chapter{Gauge transformation of the Dirac equation}
\label{chap:diracGauge}
%\useCCL
\blogpage{http://sites.google.com/site/peeterjoot/math2011/diracGauge.pdf}
\date{Aug 4, 2011}
\revisionInfo{diracGauge.tex}
\beginArtWithToc
%\beginArtNoToc
\section{Motivation}
In \citep{desai2009quantum} the gauge transformation of the Dirac equation is covered, producing the non-relativistic equation with the correct spin interaction.  There are unfortunately some sign errors, some of which self correct, and some of which do not impact the end result, but are slightly confusing.  There are also some omitted details.  I will attempt to work through the same calculation with all the signs in the right places and also fill in some of the details I found myself wanting.
%
\section{A step back.  On the gauge transformation}
%
The gauge transformations utilized are given as
%
\begin{equation}\label{eqn:diracGauge:10}
\begin{aligned}
\calE &\rightarrow \calE - e \phi \\
\Bp &\rightarrow \Bp - e \BA.
\end{aligned}
\end{equation}
%
Let us start off by reminding our self where these come from.  As outlined in \S 12.9 in \citep{jackson1975cew} (with some details pondered in \citep{phy354:hamiltonian}), our relativistic Lagrangian is
%
\begin{equation}\label{eqn:diracGauge:30}
\LL = -m c^2 \sqrt{ 1 - \frac{\Bu}{c^2}} + \frac{e}{c} \Bu \cdot \BA - e \phi.
\end{equation}
%
The conjugate momentum is
%
\begin{equation}\label{eqn:diracGauge:50}
\BP = \Be^i \PD{u^i}{\LL} = \frac{m \Bu}{\sqrt{1 - \Bu^2/c^2}} + \frac{e}{c} \BA,
\end{equation}
%
or
%
\begin{equation}\label{eqn:diracGauge:70}
\BP = \Bp + \frac{e}{c} \BA.
\end{equation}
%
The Hamiltonian, which must be expressed in terms of this conjugate momentum \(\BP\), is found to be
%
\begin{equation}\label{eqn:diracGauge:90}
\calE = \sqrt{ (c \BP - e \BA)^2 + m^2 c^4 } + e \phi.
\end{equation}
%
With the free particle Lagrangian
%
\begin{equation}\label{eqn:diracGauge:110}
\LL = -m c^2 \sqrt{ 1 - \frac{\Bu}{c^2}} ,
\end{equation}
%
our conjugate momentum is
%
\begin{equation}\label{eqn:diracGauge:130}
\BP = \frac{m \Bu}{\sqrt{ 1 - \Bu^2/c^2} }.
\end{equation}
%
For this we find that our Hamiltonian \(\calE = \BP \cdot \Bu - \LL\) is
%
\begin{equation}\label{eqn:diracGauge:150}
\calE = \frac{m c^2}{\sqrt{1 - \Bu^2/c^2}},
\end{equation}
%
but this has to be expressed in terms of \(\BP\).  Having found the form of the Hamiltonian for the interaction case, it is easily verified that \eqnref{eqn:diracGauge:90} contains the required form once the interaction fields \((\phi, \BA)\) are zeroed
%
\begin{equation}\label{eqn:diracGauge:90b}
\calE = \sqrt{ (c \BP)^2 + m^2 c^4 }.
\end{equation}
%
Considering the interaction case, Jackson points out that the energy and momentum terms can be combined as a four momentum
%
\begin{equation}\label{eqn:diracGauge:170}
p^a = \left( \inv{c}(\calE - e \phi), \BP - \frac{e}{c}\BA \right),
\end{equation}
%
so that the re-arranged and squared Hamiltonian takes the form
%
\begin{equation}\label{eqn:diracGauge:190}
p^a p_a = (m c)^2.
\end{equation}
%
From this we see that for the Lorentz force, the interaction can be found, starting with the free particle Hamiltonian \eqnref{eqn:diracGauge:90}, making the transformation
%
\begin{subequations}
\label{eqn:diracGauge:200}
\begin{equation}\label{eqn:diracGauge:700}
\begin{aligned}
\calE   &\rightarrow \calE - e\phi \\
\BP &\rightarrow \BP - \frac{e}{c}\BA,
\end{aligned}
\end{equation}
\end{subequations}
%
or in covariant form
%
\begin{equation}\label{eqn:diracGauge:210}
p^\mu \rightarrow p^\mu - \frac{e}{c}A^\mu.
\end{equation}
%
\section{On the gauge transformation of the Dirac equation}
%
The task at hand now is to make the transformations of \eqnref{eqn:diracGauge:200}, applied to the Dirac equation
%
\begin{equation}\label{eqn:diracGauge:300}
\pslash = \gamma_\mu p^\mu = m c.
\end{equation}
%
The first observation to make is that we appear to have different units in the Desai text.  Let us continue using the units from Jackson, and translate them later if inclined.
%
Right multiplication of \eqnref{eqn:diracGauge:300} by \(\gamma_0\) gives us
%
\begin{equation}\label{eqn:diracGauge:720}
\begin{aligned}
0 &= \gamma_0 (\pslash - m c) \\
  &= \gamma_0 \gamma_\mu \left( p^\mu - \frac{e}{c} A^\mu \right)
- \gamma_0 m c
\\
  &=
\gamma_0 \gamma_0 \left(\frac{\calE}{c} - \frac{e}{c} \phi \right)
+\gamma_0 \gamma_a \left(p^a - \frac{e}{c} A^a \right)
- \gamma_0 m c \\
  &=
\inv{c} \left( \calE- e \phi \right)
-\Balpha \cdot \left(\BP - \frac{e}{c} \BA \right)
- \gamma_0 m c \\
\end{aligned}
\end{equation}
%
With the minor notational freedom of using \(\gamma_0\) instead of \(\gamma_4\), this is our starting point in the Desai text, and we can now left multiply by
%
\begin{equation}\label{eqn:diracGauge:320}
(\pslash + m c) \gamma_0 =
\inv{c} \left( \calE - e \phi \right)
+\Balpha \cdot \left(\BP - \frac{e}{c} \BA \right)
+ \gamma_0 m c.
\end{equation}
%
The motivation for this appears to be that this product of conjugate like quantities
%
\begin{equation}\label{eqn:diracGauge:340}
\begin{aligned}
0 &= (\pslash + m c) \gamma_0 \gamma_0 (\pslash - m c)  \\
&=
(\pslash + m c) (\pslash - m c) \\
&= \inv{c^2} \left( \calE - e \phi \right)^2
 -\left( \BP - \frac{e}{c} \BA \right)^2 - (m c)^2 + \cdots,
\end{aligned}
\end{equation}
produces the Klein-Gordon equation, plus some cross terms to be determined.
Those cross terms are the important bits since they contain the spin interaction, even in the non-relativistic limit.
%
Let us do the expansion.
\begin{equation}\label{eqn:diracGauge:740}
\begin{aligned}
0
&= (\pslash + m c) \gamma_0 \gamma_0 (\pslash - m c) u \\
&=
\left(
\inv{c} \left( \calE - e \phi \right)
+\Balpha \cdot \left(\BP - \frac{e}{c} \BA \right)
+ \gamma_0 m c
\right)
\left(
\inv{c} \left( \calE- e \phi \right)
-\Balpha \cdot \left(\BP - \frac{e}{c} \BA \right)
- \gamma_0 m c \right) u \\
&=
\inv{c} \left( \calE - e \phi \right)
\left(
\inv{c} \left( \calE- e \phi \right)
-\Balpha \cdot \left(\BP - \frac{e}{c} \BA \right)
- \gamma_0 m c \right) u \\
&\qquad +\Balpha \cdot \left(\BP - \frac{e}{c} \BA \right)
\left(
\inv{c} \left( \calE- e \phi \right)
-\Balpha \cdot \left(\BP - \frac{e}{c} \BA \right)
- \gamma_0 m c \right) u \\
&\qquad + \gamma_0 m c
\left(
\inv{c} \left( \calE- e \phi \right)
-\Balpha \cdot \left(\BP - \frac{e}{c} \BA \right)
- \gamma_0 m c \right) u \\
&=
\left(
\inv{c^2} \left( \calE - e \phi \right)^2
- \left( \Balpha \cdot \left(\BP - \frac{e}{c} \BA \right) \right)^2
- (mc)^2
\right) u
\\
&\qquad + \inv{c} \antisymmetric{
\Balpha \cdot \left(\BP - \frac{e}{c} \BA \right)
}
{
\calE - e \phi
} u
- m c
\symmetric{\Balpha \cdot \left(\BP - \frac{e}{c} \BA \right)}{ \gamma_0} u \\
&\qquad + \cancel{\gamma_0 m
\left(
\calE - e \phi
\right) u
}
- \cancel{
\gamma_0 m
\left(
\calE - e \phi
\right) u
}
\\
\end{aligned}
\end{equation}
%
Since \(\gamma_0\) anticommutes with any \(\Balpha \cdot \Bx\), even when \(\Bx\) contains operators, the anticommutator term is killed.
%
While done in the text, lets also do the \(\Balpha \cdot \left(\BP - \frac{e}{c} \BA \right)\) square for completeness.  Because this is an operator, we need to treat this as
%
\begin{equation}\label{eqn:diracGauge:760}
\begin{aligned}
\left( \Balpha \cdot \left(\BP - \frac{e}{c} \BA \right) \right)^2 u
&=
\Balpha \cdot \left(\BP - \frac{e}{c} \BA \right)
\Balpha \cdot \left(\BP u - \frac{e}{c} \BA u \right),
\end{aligned}
\end{equation}
%
so want to treat the two vectors as independent, say \((\Balpha \cdot \Ba)(\Balpha \cdot \Bb)\).  That is
%
\begin{equation}\label{eqn:diracGauge:780}
\begin{aligned}
(\Balpha \cdot \Ba)(\Balpha \cdot \Bb)
&=
\begin{bmatrix}
0 & \Bsigma \cdot \Ba \\
\Bsigma \cdot \Ba & 0
\end{bmatrix}
\begin{bmatrix}
0 & \Bsigma \cdot \Bb \\
\Bsigma \cdot \Bb & 0
\end{bmatrix} \\
&=
\begin{bmatrix}
(\Bsigma \cdot \Ba) (\Bsigma \cdot \Bb)  & 0 \\
0 & (\Bsigma \cdot \Ba) (\Bsigma \cdot \Bb)  & 0 \\
\end{bmatrix} \\
\end{aligned}
\end{equation}
%
The diagonal elements can be expanded by coordinates
%
\begin{equation}\label{eqn:diracGauge:800}
\begin{aligned}
(\Bsigma \cdot \Ba) (\Bsigma \cdot \Bb)
&=
\sum_{m,n} \sigma^m a^m \sigma^n b^n \\
&=
\sum_m a^m b^m
+
\sum_{m\ne n} \sigma^m \sigma^n a^m b^m \\
&=
\Ba \cdot \Bb
+
i \sum_{m\ne n} \sigma^o \epsilon^{m n o} a^m b^m \\
&=
\Ba \cdot \Bb
+
i \Bsigma \cdot (\Ba \cross \Bb),
\end{aligned}
\end{equation}
%
for
%
\begin{equation}\label{eqn:diracGauge:360}
(\Balpha \cdot \Ba)(\Balpha \cdot \Bb)
=
\begin{bmatrix}
\Ba \cdot \Bb + i \Bsigma \cdot (\Ba \cross \Bb) & 0 \\
0 & \Ba \cdot \Bb + i \Bsigma \cdot (\Ba \cross \Bb)
\end{bmatrix}
\end{equation}
%
Plugging this back in, we now have an extra term in the expansion
%
\begin{equation}\label{eqn:diracGauge:820}
\begin{aligned}
0
&=
\left(
\inv{c^2} \left( \calE - e \phi \right)^2
- \left( \BP - \frac{e}{c} \BA \right)^2
- (mc)^2
\right) u
\\
&\qquad + \inv{c} \antisymmetric{
\Balpha \cdot \left(\BP - \frac{e}{c} \BA \right)
}
{
\calE - e \phi
} u
\\
&\qquad
- i \Bsigma' \cdot
\left(
\left( \BP - \frac{e}{c} \BA \right) \cross \left( \BP - \frac{e}{c} \BA \right)
\right)
 u
\end{aligned}
\end{equation}
%
Here \(\Bsigma'\) was defined as the direct product of the two by two identity with the abstract matrix \(\Bsigma\) as follows
%
\begin{equation}\label{eqn:diracGauge:380}
\Bsigma' =
\begin{bmatrix}
\Bsigma & 0 \\
0 & \Bsigma
\end{bmatrix}
= I \otimes \Bsigma
\end{equation}
%
Like the \(\BL \cross \BL\) angular momentum operator cross products this one was not zero.  Expanding it yields
%
\begin{equation}\label{eqn:diracGauge:840}
\begin{aligned}
\left( \BP - \frac{e}{c} \BA \right) \cross \left( \BP - \frac{e}{c} \BA \right)
 u
&=
\BP \cross \BP u
+ \frac{e^2}{c^2} \BA \cross \BA u
- \frac{e}{c} \left( \BA \cross \BP + \BP \cross \BA \right) u \\
&=
- \frac{e}{c} \left( \BA \cross (\BP u) + (\BP u) \cross \BA + u (\BP \cross \BA) \right) \\
&=
- \frac{e}{c} (-i \Hbar \spacegrad \cross \BA) u \\
&=
\frac{i e \Hbar}{c} \BH u
\end{aligned}
\end{equation}
%
Plugging in again we are getting closer, and now have the magnetic field cross term
%
\begin{equation}\label{eqn:diracGauge:860}
\begin{aligned}
0
&=
\left(
\inv{c^2} \left( \calE - e \phi \right)^2
- \left(\BP - \frac{e}{c} \BA \right)^2
- (mc)^2
\right) u
\\
&\qquad + \inv{c}
\antisymmetric{
\Balpha \cdot \left(\BP - \frac{e}{c} \BA \right)
}
{
\calE - e \phi
} u
\\
&\qquad
+ \frac{e \Hbar}{c} \Bsigma' \cdot \BH u.
\end{aligned}
\end{equation}
%
All that remains is evaluation of the commutator term, which should yield the electric field interaction.  That commutator is
%
\begin{equation}\label{eqn:diracGauge:880}
\begin{aligned}
\antisymmetric{
\Balpha \cdot \left(\BP - \frac{e}{c} \BA \right)
}
{
\calE - e \phi
} u
&=
\cancel{\Balpha \cdot \BP \calE u}
- e \Balpha \cdot \BP \phi u
- \frac{e}{c} \Balpha \cdot \BA \calE u
+ \cancel{\frac{e^2}{c} \Balpha \cdot \BA \phi u} \\
&
- \cancel{\calE \Balpha \cdot \BP u}
+ e \phi \Balpha \cdot \BP u
+ \frac{e}{c} \calE \Balpha \cdot \BA u
- \cancel{\frac{e^2}{c} \phi \Balpha \cdot \BA u} \\
&=
\Balpha \cdot \left( - e \BP \phi
+ \frac{e}{c} \calE \right) u \\
&=
e i \Hbar \Balpha \cdot \left( \spacegrad \phi
+ \frac{1}{c} \PD{t}{\BA} \right) u \\
&=
- e i \Hbar \Balpha \cdot \BE u
\end{aligned}
\end{equation}
%
That was the last bit required to fully expand the space time split of our squared momentum equations.  We have
%
\begin{equation}\label{eqn:diracGauge:400}
0
=
(\pslash + mc)(\pslash - mc) u
=
\left(
\inv{c^2} \left( \calE - e \phi \right)^2
- \left(\BP - \frac{e}{c} \BA \right)^2
- (mc)^2
- \frac{i e \Hbar}{c} \Balpha \cdot \BE
+ \frac{e \Hbar}{c} \Bsigma' \cdot \BH
\right) u
\end{equation}
%
This is the end result of the reduction of the spacetime split gauge transformed Dirac equation.  The next step is to obtain the non-relativistic Hamiltonian operator equation (linear in the time derivative operator and quadratic in spacial partials) that has both the electric field and magnetic field terms that we desire to accurately describe spin (actually we need only the magnetic interaction term for non-relativistic spin, but we will see that soon).
%
To obtain the first order time derivatives we can consider an approximation to the \((\calE - e \phi)^2\) terms.  We can get that by considering the difference of squares factorization
%
\begin{equation}\label{eqn:diracGauge:900}
\begin{aligned}
\inv{c^2} ( \calE - e \phi - m c^2) ( \calE - e \phi + m c^2) u
&=
\inv{c^2} \left(
( \calE - e \phi )^2 u - (m c^2)^2 u
- \cancel{m c^2 \calE u}
+ \cancel{\calE m c^2 u} \right) \\
&=
\inv{c^2} ( \calE - e \phi )^2 u - (m c)^2 u
\end{aligned}
\end{equation}
%
In the text, this is factored, instead of the factorization verified.  I wanted to be careful to ensure that the operators did not have any effect.  They do not, which is clear in retrospect since the \(\calE\) operator and the scalar \(mc\) necessarily commute.  With this factorization, some relativistic approximations are possible.  Considering the free particle energy, we can separate out the rest energy from the kinetic (which is perversely designated with subscript \(T\) for some reason in the text (and others))
%
\begin{equation}\label{eqn:diracGauge:920}
\begin{aligned}
\calE
&= \gamma m c^2  \\
&= m c^2 \left( 1 + \inv{2} \left(\frac{\Bv}{c}\right)^2 + \cdots \right) \\
&= m c^2 + \inv{2} m \Bv^2 + \cdots \\
&\equiv m c^2 + \calE_{T}
\end{aligned}
\end{equation}
%
With this definition, the energy minus mass term in terms of kinetic energy (that we also had in the Klein-Gordon equation) takes the form
%
\begin{equation}\label{eqn:diracGauge:420}
\inv{c^2} ( \calE - e \phi )^2 u - (m c)^2 u
=
\inv{c^2} ( \calE_{T} - e \phi ) ( \calE - e \phi + m c^2) u
\end{equation}
%
In the second factor, to get a non-relativistic approximation of \(\calE - e \phi\), the text states without motivation that \(e \phi\) will be considered small compared to \(m c^2\).  We can make some sense of this by considering the classical Hamiltonian for a particle in a field
%
\begin{equation}\label{eqn:diracGauge:940}
\begin{aligned}
\calE
&= \sqrt{ c^2 \left(\BP - \frac{e}{c} \BA\right) + (m c^2)^2 } + e \phi \\
&= \sqrt{ c^2 (\gamma m \Bv)^2 + (m c^2)^2 } + e \phi \\
&= m c \sqrt{ (\gamma \Bv)^2 + c^2 } + e \phi \\
&= m c \sqrt{ \frac{ \Bv^2 + c^2 ( 1 - \Bv^2/c^2) } { 1 - \Bv^2/c^2 } } + e \phi \\
&= \gamma m c^2 + e \phi \\
&= m c^2 \left( 1 + \inv{2} \frac{\Bv^2}{c^2} + \cdots \right) + e \phi.
\end{aligned}
\end{equation}
%
We find that, in the non-relativistic limit, we have
%
\begin{equation}\label{eqn:diracGauge:440}
\calE - e \phi = m c^2 + \inv{2} m \Bv^2 + \cdots \approx m c^2,
\end{equation}
%
and obtain the first order approximation of our time derivative operator
%
\begin{equation}\label{eqn:diracGauge:460}
\inv{c^2} ( \calE - e \phi )^2 u - (m c)^2 u
\approx
\inv{c^2} ( \calE_{T} - e \phi ) 2 m c^2 u,
\end{equation}
%
or
\begin{equation}\label{eqn:diracGauge:480}
\inv{c^2} ( \calE - e \phi )^2 u - (m c)^2 u
\approx
2 m ( \calE_{T} - e \phi ).
\end{equation}
%
It seems slightly underhanded to use the free particle Hamiltonian in one part of the approximation, and the Hamiltonian for a particle in a field for the other part.  This is probably why the text just mandates that \(e\phi\) be small compared to \(m c^2\).
%
To summarize once more before the final reduction (where we eliminate the electric field component of the operator equation), we have
%
\begin{equation}\label{eqn:diracGauge:400b}
0
=
(\pslash + mc)(\pslash - mc) u
\approx
\left(
2 m ( \calE_{T} - e \phi )
- \left(\BP - \frac{e}{c} \BA \right)^2
- \frac{i e \Hbar}{c} \Balpha \cdot \BE
+ \frac{e \Hbar}{c} \Bsigma' \cdot \BH
\right) u.
\end{equation}
%
Except for the electric field term, this is the result that is derived in the text.  It was argued that this term is not significant compared to \(e \phi\) when the particle velocity is restricted to the non-relativistic domain.  This is done by computing the expectation of this term relative to \(e \phi\).  Consider
%
\begin{equation}\label{eqn:diracGauge:500}
\Abs{ \expectation{ \frac{e \Hbar}{ 2 m c} \frac{\Balpha \cdot \BE}{e \phi } } }
\end{equation}
%
With the velocities low enough so that the time variation of the vector potential does not contribute to the electric field (i.e. the electrostatic case), we have
%
\begin{equation}\label{eqn:diracGauge:520}
\BE = - \spacegrad \phi = - \rcap \PD{r}{\phi}.
\end{equation}
%
The variation in length \(a\) that is considered is labeled the characteristic length
%
\begin{equation}\label{eqn:diracGauge:540}
p a \sim \Hbar,
\end{equation}
%
so that with \(p = m v\) we have
%
\begin{equation}\label{eqn:diracGauge:560}
a \sim \frac{\Hbar}{m v}.
\end{equation}
%
This characteristic length is not elaborated on, but one can observe the similarity to the Compton wavelength
%
\begin{equation}\label{eqn:diracGauge:580}
L_{\text{Compton}} = \frac{\Hbar}{m c},
\end{equation}
%
the length scale for which Quantum field theory must be considered.  This length scale is considerably larger for velocities smaller than the speed of light.  For example, the drift velocity of electrons in copper is \(\sim 10^{6} \frac{\text{m}}{\text{s}}\), which fixes our length scale to \(100\) times the Compton length (\(\sim 10^{-12} \text{m})\).  This is still a very small length, but is in the QM domain instead of QED.  With such a length scale consider the magnitude of a differential contribution to the electric field
%
\begin{equation}\label{eqn:diracGauge:600}
\Abs{\phi} = \Abs{\BE} \Delta x = \Abs{\BE} a,
\end{equation}
%
so that
%
\begin{equation}\label{eqn:diracGauge:960}
\begin{aligned}
\expectation{ \frac{e \Hbar}{ 2 m c} \frac{\Balpha \cdot \BE}{e \Abs{\phi} } }
&=
\expectation{ \frac{e \Hbar}{ 2 m c} \frac{\Balpha \cdot \BE}{e a \Abs{\BE} } } \\
&=
\expectation{ \frac{e \Hbar}{m} \frac{1}{ 2 c} \frac{\Balpha \cdot \BE}{e \frac{\Hbar }{ m v } \Abs{\BE} } } \\
&=
\inv{2} \frac{v}{c} \expectation{ \frac{\Balpha \cdot \BE}{ \Abs{\BE} } }.
\end{aligned}
\end{equation}
%
Thus the magnitude of this (vector) expectation is dominated by the expectation of just the \(\Balpha\).  That has been calculated earlier when Dirac currents were considered, where it was found that
%
\begin{equation}\label{eqn:diracGauge:620}
\expectation{\alpha_i} = \psi^\dagger \alpha_i \psi = (\Bj)_i.
\end{equation}
%
Also recall that (33.73) that this current was related to momentum with
%
\begin{equation}\label{eqn:diracGauge:640}
\Bj = \frac{\Bp}{m c} = \frac{\Bv}{c}
\end{equation}
%
which allows for a final approximation of the magnitude of the electric field term's expectation value relative to the \(e\phi\) term of the Hamiltonian operator.  Namely
%
\begin{equation}\label{eqn:diracGauge:660}
\Abs{ \expectation{ \frac{e \Hbar}{ 2 m c} \frac{\Balpha \cdot \BE}{e \phi } } }
\sim
\frac{\Bv^2}{c^2}.
\end{equation}
%
With that last approximation made, the gauge transformed Dirac equation, after non-relativistic approximation of the energy and electric field terms, is left as
%
\begin{equation}\label{eqn:diracGauge:680}
i \Hbar \PD{t}{}
=
\inv{2m} \left(i \Hbar \spacegrad + \frac{e}{c} \BA \right)^2
- \frac{e \Hbar}{2 m c} \Bsigma' \cdot \BH
+ e \phi.
\end{equation}
%
This is still a four dimensional equation, and it is stated in the text that only the large component is relevant (reducing the degrees of spin freedom to two).  That argument makes a bit more sense with the matrix form of the gauge reduction which follows in the next section, so understanding that well seems worthwhile, and is the next thing to digest.
\EndArticle

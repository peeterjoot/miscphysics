%
% Copyright � 2012 Peeter Joot.  All Rights Reserved.
% Licenced as described in the file LICENSE under the root directory of this GIT repository.
%

%
%
%\documentclass{article}

%\input{../peeters_macros.tex}
%\input{../peeters_macros2.tex}

%\usepackage[bookmarks=true]{hyperref}

%\usepackage{color,cite,graphicx}
   % use colour in the document, put your citations as [1-4]
   % rather than [1,2,3,4] (it looks nicer, and the extended LaTeX2e
   % graphics package.
%\usepackage{latexsym,amssymb,epsf} % do not remember if these are
   % needed, but their inclusion can not do any damage

\chapter{Ad-hoc motivation of some QM wave equations}
\label{chap:sch}
%\author{Peeter Joot \quad peeterjoot@protonmail.com}
\date{ Dec 13, 2008.  sch.tex }

%\begin{document}

%\maketitle{}

%\tableofcontents

\section{Motivation}

\subsection{Non-Relativistic case}

A common (cf: wikipedia and \citep{french1998iqp}) introductory motivation for the non-relativistic Schr\"{o}dinger's equation appears to follow the following lines.  Assume that
we desire a plane wave equation of the following form

\begin{equation}\label{eqn:sch:20}
\begin{aligned}
\psi = \exp(i (\Bk \cdot \Bx - \omega t))
\end{aligned}
\end{equation}

plus a requirement that we have a total energy that can be expressed in terms of kinetic plus potential

\begin{equation}\label{eqn:sch:energy}
\begin{aligned}
E = \frac{\Bp^2}{2m} + V
\end{aligned}
\end{equation}

We also have the Einstein relationship

\begin{equation}\label{eqn:sch:40}
\begin{aligned}
E = h \nu = \frac{h}{2\pi} 2 \pi \nu = \Hbar \omega
\end{aligned}
\end{equation}

and the DeBroglie Hypothesis for the magnitude of the momentum of a particle

\begin{equation}\label{eqn:sch:60}
\begin{aligned}
p = \frac{h}{\lambda}.
\end{aligned}
\end{equation}

In terms of wave number \(2 \pi k = \inv{\lambda}\) this last is

\begin{equation}\label{eqn:sch:80}
\begin{aligned}
p = \frac{h}{2\pi} k = \Hbar k.
\end{aligned}
\end{equation}

or in three dimensions

\begin{equation}\label{eqn:sch:100}
\begin{aligned}
\Bp = \Hbar \Bk.
\end{aligned}
\end{equation}

Taking derivatives of the postulated wave function one has

\begin{equation}\label{eqn:sch:timepartial}
\begin{aligned}
\frac{\partial \psi}{\partial t} = -i \omega \psi
\end{aligned}
\end{equation}

and
\begin{equation}\label{eqn:sch:laplacian}
\begin{aligned}
\spacegrad^2 \psi = i^2 \sum_j k_j^2 \psi = - \Bk^2 \psi
\end{aligned}
\end{equation}

From the energy relationship, if one requires that
\begin{equation}\label{eqn:sch:120}
\begin{aligned}
E \psi &= \left(\frac{\Bp^2}{2m} + V\right) \psi \\
\Hbar \omega \psi &= \left(\Hbar^2 \frac{\Bk^2}{2m} + V\right) \psi \\
\end{aligned}
\end{equation}

and then substituting the derivatives from equations \eqnref{eqn:sch:timepartial} and \eqnref{eqn:sch:laplacian} we have

\begin{equation}\label{eqn:sch:140}
\begin{aligned}
i \Hbar \PD{t}{\psi} &= \left(-\frac{\Hbar^2}{2m} \spacegrad^2  + V\right) \psi \\
\end{aligned}
\end{equation}

\subsection{Relativistic case}

The relativistic force free Schr\"{o}dinger's equation is motivated by \citep{srednicki2007qft} replacing the Hamiltonian operator \(H = \BP^2/{2 m}\) with

\begin{equation}\label{eqn:sch:160}
\begin{aligned}
H = \sqrt{m^2 c^4 + \BP^2 c^2} \approx m c^2 + \BP^2/2m
\end{aligned}
\end{equation}

for

\begin{equation}\label{eqn:sch:180}
\begin{aligned}
i \Hbar \PD{t}{\psi} = \sqrt{ -\Hbar^2 c^2 \spacegrad^2 + m^2 c^4} \psi
\end{aligned}
\end{equation}

then squaring the operators on both sides, removing the root:

\begin{equation}\label{eqn:sch:200}
\begin{aligned}
- \Hbar^2 \PDsq{x^0}{\psi} &= \left(-\Hbar^2 \spacegrad^2 + m^2 c^2 \right) \psi \\
- \Hbar^2 \PDsq{x^0}{\psi} + \Hbar^2 \spacegrad^2 &= m^2 c^2 \psi \\
\end{aligned}
\end{equation}

Which is called the Klein-Gordon equation:
\begin{equation}\label{eqn:sch:220}
\begin{aligned}
\left(\frac{\Hbar^2}{2m} \grad^2 + \inv{2} m c^2\right) \psi = 0 \\
\end{aligned}
\end{equation}

The Noether's theorem wikipedia article, and Klein-Gordon pages give the action
as (removing use of natural units and changing to a \(+---\) metric) gives:

\begin{equation}\label{eqn:sch:240}
\begin{aligned}
\LL &= -\eta^{\mu\nu} \partial_\mu \psi \partial_\nu \psi^\conj + \frac{m^2 c^2}{\Hbar^2} \psi \psi^\conj \\
&= -\partial^\nu \psi \partial_\nu \psi^\conj + \frac{m^2 c^2}{\Hbar^2} \psi \psi^\conj \\
\end{aligned}
\end{equation}

This first term is a squared spacetime gradient

\begin{equation}\label{eqn:sch:260}
\begin{aligned}
(\grad \psi) \cdot (\grad \psi^\conj)
&= (\gamma^\mu \partial_\mu \psi) \cdot (\gamma_\nu \partial^\nu \psi^\conj) \\
&= {\delta^\mu}_\nu \partial_\mu \psi \partial^\nu \psi^\conj \\
&= \partial_\mu \psi \partial^\mu \psi^\conj \\
\end{aligned}
\end{equation}

so we have
\begin{equation}\label{eqn:sch:280}
\begin{aligned}
\LL
&= -\eta^{\mu\nu} \partial_\mu \psi \partial_\nu \psi^\conj + \frac{m^2 c^2}{\Hbar^2} \psi \psi^\conj \\
&= -(\grad \psi) \cdot (\grad \psi^\conj) + \frac{m^2 c^2}{\Hbar^2} \psi \psi^\conj \\
\end{aligned}
\end{equation}

So, here we expect to get a spacetime Laplacian, like the Maxwell potential equation, and one does:

\begin{equation}\label{eqn:sch:300}
\begin{aligned}
-\grad^2 \psi = \frac{m^2 c^2}{\Hbar^2} \psi
\end{aligned}
\end{equation}

Now, the Srednicki text indicates that Dirac linearized this equation, to get back something that was
first order in the time derivative.

I do not yet follow that description, but see that a linearization is possible by taking roots of the operators above,
undoing the somewhat fishy seeming squaring done to arrive at the Klein-Gordon equation in the first place

\begin{equation}\label{eqn:sch:dirac}
\begin{aligned}
i \Hbar \grad \psi = \pm m c \psi
\end{aligned}
\end{equation}

Is this equivalent to Dirac's formulation?  Comparison to
\href{https://en.wikipedia.org/wiki/Dirac_equation#Covariant_form_and_relativistic_invariance}, where the Dirac equation is given in covariant
form

\begin{equation}\label{eqn:sch:320}
\begin{aligned}
i \Hbar \gamma^\mu \partial_\mu \psi - m c \psi = 0
\end{aligned}
\end{equation}

which is the positive variant of \eqnref{eqn:sch:dirac}.

What is the Lagrangian that is associated with this?  The most probable interpretation of \(i\) here is the Minkowski pseudoscalar, as opposed to
a unit bivector of spacetime vectors or some other geometrical object with \(-1\) square.

%\bibliographystyle{plainnat}
%\bibliography{myrefs}

%\end{document}

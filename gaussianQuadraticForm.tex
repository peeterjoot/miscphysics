%
% Copyright � 2013 Peeter Joot.  All Rights Reserved.
% Licenced as described in the file LICENSE under the root directory of this GIT repository.
%
%\input{../blogpost.tex}
%\renewcommand{\basename}{gaussianQuadraticForm}
%\renewcommand{\dirname}{notes/miscphysics/}
%%\newcommand{\dateintitle}{}
%\newcommand{\keywords}{Gaussian, quadratic form, exponential integral}
%
%\input{../peeter_prologue_print2.tex}
%
%\beginArtNoToc
%
%\generatetitle{Gaussian quadratic form integrals and multivariable approximation of exponential integrals}
\chapter{Gaussian quadratic form integrals and multivariable approximation of exponential integrals}
\label{chap:gaussianQuadraticForm}
\section{Motivation}
In \citep{zee2005quantum} \texteqnref{I.2.20} is the approximation
\begin{equation}\label{eqn:gaussianQuadraticForm:20}
\int d\Bq e^{-f(\Bq)/\Hbar} \approx e^{-f(\Ba)/\Hbar} \sqrt{
\frac{2 \pi \Hbar}{\Det f''(\Ba)} },
\end{equation}
where \([ f''(\Ba) ]_{ij} \equiv \evalbar{\partial^2 f/\partial q_i \partial q_j}{\Bq = \Ba}\).  Here \(\Ba\) is assumed to be an extremum of \(f\).  This follows from a generalization of the Gaussian integral result.  Let's derive both in detail.
\section{Guts}
First, to second order, let's expand \(f(\Bq)\) around a min or max at \(\Bq = \Ba\).  The usual trick, presuming that one doesn't remember the form of this generalized Taylor expansion, is to expand \(g(t) = f(\Ba + t \Bq)\) around \(t = 0\), then evaluate at \(t = 1\).  We have
\begin{equation}\label{eqn:gaussianQuadraticForm:40}
g'(t)
= \sum_i \PD{(a_i + t q_i)}{f(\Ba + t \Bq)} \ddt{ (a_i + t q_i) }
= \sum_i q_i \PD{(a_i + t q_i)}{f(\Ba + t \Bq)}
= \Bq \cdot
\lr{ \evalbar{\spacegrad_\Bq f(\Bq)}{\Bq = \Ba + t \Bq} }.
\end{equation}

The second derivative is
\begin{equation}\label{eqn:gaussianQuadraticForm:60}
g''(t)
= \sum_{i j} q_i q_j \PD{(a_j + t q_j)}{} \PD{(a_i + t q_i)}{f(\Ba + t \Bq)},
\end{equation}

This gives
\begin{equation}\label{eqn:gaussianQuadraticForm:80}
\begin{aligned}
g'(0) &= \Bq \cdot \spacegrad_\Bq f(\Bq) = \sum_i q_i \partial q_i f(\Bq) \\
g''(0) &=
\lr{\Bq \cdot \spacegrad_\Bq}^2
f(\Bq) = \sum_{i j} q_i q_j \partial q_i \partial q_j f(\Bq).
\end{aligned}
\end{equation}

Putting these together, we have to second order in \(t\) is
\begin{equation}\label{eqn:gaussianQuadraticForm:100}
f(\Ba + t \Bq) \approx f(\Ba)
+
\sum_i q_i \partial q_i f(\Bq) \frac{t^1}{1!}
+
\sum_{i j} q_i q_j \partial q_i \partial q_j f(\Bq) \frac{t^2}{2!},
\end{equation}
or
\begin{equation}\label{eqn:gaussianQuadraticForm:120}
f(\Ba + \Bq) \approx f(\Ba)
+
\sum_i q_i \evalbar{
\lr{ \PD{q_i}{f}}
}{\Ba}
+
\inv{2} \sum_{i j} q_i q_j
\evalbar{
\lr{ \frac{\partial^2 f}{\partial q_i \partial q_j} }
}{\Ba}.
\end{equation}

We can put the terms up to second order in a nice tidy matrix forms
\begin{subequations}
\label{eqn:gaussianQuadraticForm:320a}
\begin{equation}\label{eqn:gaussianQuadraticForm:320}
\Bb = \evalbar{
\lr{ \spacegrad_\Bq f}
}{\Ba}
\end{equation}
\begin{equation}\label{eqn:gaussianQuadraticForm:160}
A =
{\begin{bmatrix}
\evalbar{
\lr{ \frac{\partial^2 f}{\partial q_i \partial q_j}}
}{\Ba}
\end{bmatrix}}_{i j}.
\end{equation}
\end{subequations}

Note that \eqnref{eqn:gaussianQuadraticForm:160} is a real symmetric matrix, and can thus be reduced to diagonal form by an orthonormal transformation.  Putting the pieces together, we have
\begin{equation}\label{eqn:gaussianQuadraticForm:180}
f(\Ba + \Bq) \approx f(\Ba) + \Bq^\T \Bb + \inv{2} \Bq^\T A \Bq.
\end{equation}

Integrating this, we have
\begin{equation}\label{eqn:gaussianQuadraticForm:200}
\int dq_1 dq_2 \cdots dq_N \exp
\left( -
\lr{ f(\Ba) + \Bq^\T \Bb + \inv{2} \Bq^\T A \Bq }
\right)
=
e^{-f(\Ba)}
\int dq_1 dq_2 \cdots dq_N \exp
\lr{ -\Bq^\T \Bb - \inv{2} \Bq^\T A \Bq }.
\end{equation}

Employing an orthonormal change of variables to diagonalize the matrix

\begin{equation}\label{eqn:gaussianQuadraticForm:220}
A = O^\T D O,
\end{equation}
and \(\Br = O \Bq\), or \(r_i = O_{ik} q_k\), the volume element after transformation is
\begin{equation}\label{eqn:gaussianQuadraticForm:240}
\begin{aligned}
dr_1
dr_2
\cdots
dr_N
&=
\frac
{\partial(r_1, r_2, \cdots, r_N)}
{\partial(q_1, q_2, \cdots, q_N)}
dq_1
dq_2
\cdots
dq_N \\
&=
\begin{vmatrix}
O_{11} & O_{12} & \cdots & O_{1N} \\
O_{21} & O_{22} & \cdots & O_{2N} \\
\vdots & \vdots & \vdots & \vdots \\
O_{N1} & O_{N2} & \cdots & O_{NN} \\
\end{vmatrix}
dq_1
dq_2
\cdots
dq_N \\
&=
(\Det O)
dq_1
dq_2
\cdots
dq_N \\
&=
dq_1
dq_2
\cdots
dq_N
\end{aligned}
\end{equation}

Our integral is
\begin{equation}\label{eqn:gaussianQuadraticForm:260}
\begin{aligned}
e^{-f(\Ba)}
\int dq_1 dq_2 \cdots dq_N \exp
\lr{ -\Bq^\T \Bb - \inv{2} \Bq^\T A \Bq }
&=
e^{-f(\Ba)}
\int dr_1 dr_2 \cdots dr_N \exp
\lr{ -\Bq^\T O^\T O \Bb - \inv{2} \Bq^\T O^\T D O \Bq } \\
&=
e^{-f(\Ba)}
\int dr_1 dr_2 \cdots dr_N \exp
\lr{ -\Br^\T (O \Bb) - \inv{2} \Br^\T D \Br } \\
&=
e^{-f(\Ba)}
\int dr_1 e^{ -\inv{2} r_1^2 \lambda_1 - r_1 (O \Bb)_1 }
\int dr_2 e^{ -\inv{2} r_2^2 \lambda_2 - r_2 (O \Bb)_2 }
\cdots
\int dr_N e^{ -\inv{2} r_N^2 \lambda_N - r_N (O \Bb)_N }.
\end{aligned}
\end{equation}

We now have products of terms that are of the regular Gaussian form.  One such integral is
\begin{equation}\label{eqn:gaussianQuadraticForm:280}
\begin{aligned}
\int e^{-a x^2/2 + J x}
&=
\int \exp
\left(
-\inv{2}
\left(
\lr{ \sqrt{a} x - J/\sqrt{a} }^2
-
\lr{ J/\sqrt{a} }^2
\right)
\right) \\
&=
e^{J^2/2a}
\sqrt{
2 \pi \int_0^\infty r dr e^{-a r^2/2}
}
\end{aligned}
\end{equation}

This is just
\begin{equation}\label{eqn:gaussianQuadraticForm:300}
\int e^{-a x^2/2 + J x}
=
e^{J^2/2a} \sqrt{ \frac{2 \pi}{a} }.
\end{equation}

Applying this to the integral of interest, writing \(m_i = (O \Bb)_i\)
\begin{equation}\label{eqn:gaussianQuadraticForm:260}
\begin{aligned}
e^{-f(\Ba)}
\int &dq_1 dq_2 \cdots dq_N \exp
\lr{ -\Bq^\T \Bb - \inv{2} \Bq^\T A \Bq } \\
&=
e^{-f(\Ba)}
e^{-m_1^2/2\lambda_1} \sqrt{ \frac{2 \pi}{\lambda_1}}
e^{-m_2^2/2\lambda_2} \sqrt{ \frac{2 \pi}{\lambda_2}}
\cdots
e^{-m_N^2/2\lambda_N} \sqrt{ \frac{2 \pi}{\lambda_N}} \\
&=
e^{-f(\Ba)}
\sqrt{
\frac{2 \pi}{\Det A}
}
\exp
\left(
-\inv{2}
\lr{
-m_1^2/\lambda_1
-m_2^2/\lambda_2
\cdots
-m_N^2/\lambda_N
}
\right).
\end{aligned}
\end{equation}

This last exponential argument can be put into matrix form
\begin{equation}\label{eqn:gaussianQuadraticForm:340}
\begin{aligned}
-m_1^2/\lambda_1
-m_2^2/\lambda_2
\cdots
-m_N^2/\lambda_N
&= (O \Bb)^\T D^{-1} O \Bb \\
&= \Bb^\T O^\T D^{-1} O \Bb \\
&= \Bb^\T A^{-1} \Bb,
\end{aligned}
\end{equation}

Finally, referring back to \eqnref{eqn:gaussianQuadraticForm:320a}, we have
\begin{equation}\label{eqn:gaussianQuadraticForm:360}
\int d\Bq e^{-f(\Bq)} \approx e^{-f(\Ba)}
\sqrt{
\frac{2 \pi}{\Det A}
}
e^{-\Bb^\T A^{-1} \Bb/2}.
\end{equation}

Observe that we can recover \eqnref{eqn:gaussianQuadraticForm:20} by noting that \(\Bb = 0\) for that system was assumed (i.e. \(\Ba\) was an extremum point), and by noting that the determinant scales with \(1/\Hbar\) since it just contains the second partials.
\paragraph{An afterword on notational sugar:}
We didn't need it, but it seems worth noting that we can write the Taylor expansion of \eqnref{eqn:gaussianQuadraticForm:180} in operator form as
\begin{equation}\label{eqn:gaussianQuadraticForm:140}
f(\Ba + \Bq) =
\sum_{k = 0}^\infty \inv{k!}
\evalbar{
\lr{\Bq \cdot \spacegrad_{\Bq'}}^k
 f(\Bq') }{\Bq' = \Ba}
=
\evalbar{ e^{\Bq \cdot \spacegrad_{\Bq'}} f(\Bq') }{\Bq' = \Ba}.
\end{equation}
%\EndArticle

%
% Copyright � 2012 Peeter Joot.  All Rights Reserved.
% Licenced as described in the file LICENSE under the root directory of this GIT repository.
%

%
%
%\documentclass{article}

%\input{../peeters_macros.tex}
%\input{../peeters_macros2.tex}
%\usepackage[bookmarks=true]{hyperref}

%\usepackage{color,cite,graphicx}
   % use colour in the document, put your citations as [1-4]
   % rather than [1,2,3,4] (it looks nicer, and the extended LaTeX2e
   % graphics package.
%\usepackage{latexsym,amssymb,epsf} % do not remember if these are
   % needed, but their inclusion can not do any damage


\chapter{Pauli's relativity background in QM intro from "Wave Mechanics"}
\label{chap:pauliQmRelativityIntro}
%\author{Peeter Joot \quad peeterjoot@protonmail.com}
\date{ Jan 24, 2009.  pauliQmRelativityIntro.tex }

%\begin{document}

%\maketitle{}

%\tableofcontents

\section{Motivation}

In \citep{pauli2000wm} a few relativity notes are made to build up to
a relativistic wave equation (ie: the Klein-Gordon equation),
and show one can introduce a non-relativistic approximation of this
that has close to the form of a the free particle \Sch equation.  It
is interesting to see things use relativity as a base.  This is exactly
opposite to the Klein-Gordon treatment in a text such as
\citep{srednicki2007qft} where a way to find a
relativistically correct form starting from the \Sch equation is searched for.

Pauli's treatment is a bit too terse for me, but has a number of
interesting and illuminating features.  Here I walk through his treatment
at my own pace.

\section{Relativistic mechanics}

\subsection{Energy in terms of momentum}

Equation \(1.4\) is the famous energy and momentum equations

\begin{equation}\label{eqn:pauli_qm_relativity_intro:p1_4}
\begin{aligned}
E &= \frac{ m c^2 }{\sqrt{1 - \Bv^2/c^2}} \\
\Bp &= \frac{ m \Bv }{\sqrt{1 - \Bv^2/c^2}} \\
\end{aligned}
\end{equation}

These pair of these quantities is often now expressed as a four vector in various ways

\begin{equation}\label{eqn:pauliQmRelativityIntro:21}
\begin{aligned}
p &= \left(\frac{E}{c}, \Bp \right) \\
p &= \frac{E}{c} \gamma_0 + \Bp \gamma_0 \\
\cdots
\end{aligned}
\end{equation}

These two quantities are observably interdependent, and this dependency can be made explicit by forming the sum

\begin{equation}\label{eqn:pauliQmRelativityIntro:41}
\begin{aligned}
\Bp^2 + m^2 c^2
&= \frac{ m^2 \Bv^2 }{{1 - \Bv^2/c^2}} +  \frac{ m^2 c^2( 1 - \Bv^2/c^2) }{{1 - \Bv^2/c^2}}  \\
&= \inv{1 - \Bv^2/c^2} m^2 \left( \Bv^2 + c^2( 1 - \Bv^2/c^2) \right) \\
&= \inv{1 - \Bv^2/c^2} m^2 \left( \Bv^2 + c^2 - \Bv^2 \right) \\
&= \inv{1 - \Bv^2/c^2} m^2 c^2 \\
\end{aligned}
\end{equation}

This recovers Pauli's equation \(1.3\) (in the square).

\begin{equation}\label{eqn:pauli_qm_relativity_intro:p1_3}
\begin{aligned}
\frac{E^2}{c^2} = \Bp^2 + m^2 c^2
\end{aligned}
\end{equation}

This is slightly different from how I am used to seeing this expressed, since Energy is singled out.
Rearranging slightly recovers the scalar invariant for the energy momentum four vector:

\begin{equation}\label{eqn:pauliQmRelativityIntro:61}
\begin{aligned}
m^2 c^2 &= \frac{E^2}{c^2} -\Bp^2
\end{aligned}
\end{equation}

\subsection{Energy-momentum four vector from Energy}

Now, interestingly, Pauli also points out that his \eqnref{eqn:pauli_qm_relativity_intro:p1_3} can be used to derive the four vector
equations for energy and momentum, only requiring one express the relationship between Kinetic energy and momentum
as one would do in plain old non-relativistic physics.  That is, starting with

\begin{equation}\label{eqn:pauliQmRelativityIntro:81}
\begin{aligned}
E &= \inv{2} m \Bv^2 \\
\end{aligned}
\end{equation}

differentiation with respect to some parameter we can write

\begin{equation}\label{eqn:pauliQmRelativityIntro:101}
\begin{aligned}
\frac{dE}{d\alpha}
&= m \Bv \cdot \frac{d\Bv}{d\alpha} \\
&= \Bv \cdot \frac{d\Bp}{d\alpha} \\
\end{aligned}
\end{equation}

If the specific parametrization of the path is implied we have

\begin{equation}\label{eqn:pauli_qm_relativity_intro:EvP}
\begin{aligned}
{dE} &= \Bv \cdot {d\Bp}.
\end{aligned}
\end{equation}

In coordinates this gives

\begin{equation}\label{eqn:pauliQmRelativityIntro:121}
\begin{aligned}
{dE} &= \sum_k v_k dp_k \\
\end{aligned}
\end{equation}

Pauli uses this to express the velocity coordinates in terms of energy and momentum, and writes

\begin{equation}\label{eqn:pauli_qm_relativity_intro:vEp}
\begin{aligned}
v_k &= \PD{p_k}{E}
\end{aligned}
\end{equation}

My way of getting this seems a bit fishy, dropping the explicit parametrization to get the one form, and then switching magically to partials, but once one gets to the end result it does not appear unreasonable.

Perhaps better is to skip the one form business completely, writing

\begin{equation}\label{eqn:pauliQmRelativityIntro:141}
\begin{aligned}
E = \inv{2} \Bv \cdot \Bp = \inv{2} \sum_k v_k p_k
\end{aligned}
\end{equation}

But taking partials from this to get \eqnref{eqn:pauli_qm_relativity_intro:vEp} requires care since \(p_k\) and \(v_k\) are dependent.
%Perhaps notable is that Pauli goes from \eqnref{eqn:pauli_qm_relativity_intro:EvP} to \eqnref{eqn:pauli_qm_relativity_intro:vEp} directly.

Assuming \eqnref{eqn:pauli_qm_relativity_intro:vEp} is valid and applying this to \eqnref{eqn:pauli_qm_relativity_intro:p1_3}, it is relatively straightforward to
recover the four-vector energy-momentum equations.

From
\begin{equation}\label{eqn:pauliQmRelativityIntro:161}
\begin{aligned}
E &= \sqrt{\Bp^2 c^2 + m^2 c^4 } \\
\end{aligned}
\end{equation}

we calculate
\begin{equation}\label{eqn:pauliQmRelativityIntro:181}
\begin{aligned}
v_k
&= \PD{p_k}{E} \\
&= (2 p_k c^2) \inv{2} \frac{1}{\sqrt{\Bp^2 c^2 + m^2 c^4 }} \\
&= c^2 \frac{p_k}{E}
\end{aligned}
\end{equation}

Summing over all components
\begin{equation}\label{eqn:pauliQmRelativityIntro:201}
\begin{aligned}
\frac{\Bv^2}{c^2}
&= \sum_k \frac{(v_k)^2}{c^2} \\
&= c^2 \sum_k \frac{{p_k}^2}{E^2} \\
&= c^2 \frac{\Bp^2}{E^2} \\
\end{aligned}
\end{equation}

Subtracting this from one, gives us our gamma factor (squared), which is

\begin{equation}\label{eqn:pauliQmRelativityIntro:221}
\begin{aligned}
1 - \frac{\Bv^2}{c^2}
&= 1 - c^2 \frac{\Bp^2}{E^2} \\
&= \inv{E^2} \left( \Bp^2 c^2 + m^2 c^4  - c^2 {\Bp^2} \right) \\
&= \frac{m^2 c^4}{E^2} \\
\end{aligned}
\end{equation}

So, we have the energy half of \eqnref{eqn:pauli_qm_relativity_intro:p1_4}

\begin{equation}\label{eqn:pauliQmRelativityIntro:241}
\begin{aligned}
E^2 &= \frac{m^2 c^4}{1 - \frac{\Bv^2}{c^2} }
\end{aligned}
\end{equation}

For the momentum we then have
\begin{equation}\label{eqn:pauliQmRelativityIntro:261}
\begin{aligned}
\Bp^2 c^2 + m^2 c^4 &= \frac{m^2 c^4}{1 - \frac{\Bv^2}{c^2} }
\end{aligned}
\end{equation}
\begin{equation}\label{eqn:pauliQmRelativityIntro:281}
\begin{aligned}
\Bp^2
&= \frac{m^2 c^2}{1 - \frac{\Bv^2}{c^2} } - m^2 c^2 \frac{(1 - \frac{\Bv^2}{c^2} )}{1 - \frac{\Bv^2}{c^2} } \\
&= \frac{m^2 \Bv^2}{1 - \frac{\Bv^2}{c^2} }
\end{aligned}
\end{equation}

the second half of \eqnref{eqn:pauli_qm_relativity_intro:p1_4}.

Pretty cool.  Given the energy momentum invariant, \(m^2 c^2 = E^2/c^2 - \Bp^2\), and a requirement that the velocity, momentum, Kinetic energy combination is related precisely as in classical mechanics, with \(v_k = \PDi{p_k}{E}\), we
recover the relativistic energy momentum four vector.

This is probably not surprising to somebody who knows relativity better than I, but it was
interesting to me to see this worked ``backwards'' this way.

\subsection{After note}

A timely listening to Susskind's classical mechanics lecture 6, shows that
this surprising method used by Pauli to work backwards from the energy
is in fact a use of the Hamiltonian formalism to relate energy, velocity
and position.  We see here that one logically just has to pick the ``right''
energy construct, then the familiar relativistic energy and momentum relations
follow directly.  This requires nothing more than using the Hamiltonian
relationships in the same way that we would get the Newtonian equations
of motion from a classical energy relationship.

My failure to study the Hamiltonian formalism now stands out.  I planned to
get to it eventually in a QM context, but Pauli shows here that an understanding
of that tool set is well justified in a classical mechanics context as well.

%\bibliographystyle{plainnat}
%\bibliography{myrefs}

%\end{document}

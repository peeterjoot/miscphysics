%
% Copyright � 2012 Peeter Joot.  All Rights Reserved.
% Licenced as described in the file LICENSE under the root directory of this GIT repository.
%
% pick one:
%\input{../assignment.tex}
%\input{../blogpost.tex}
%\renewcommand{\basename}{divisibleByThree}
%\renewcommand{\dirname}{notes/miscphysics/}
%\newcommand{\dateintitle}{}
%\newcommand{\keywords}{}
%
%\input{../peeter_prologue_print2.tex}
%
%\beginArtNoToc
%
%\generatetitle{A condition for divisibility by three}
\chapter{A condition for divisibility by three}
%\label{chap:\basename}
\section{Motivation}
My daughter told me that any number that any number who's digits add up to a multiple of three are divisible by three.  I'd never heard of such a thing, and was surprised by it.  In the spirit of a true geek dad, I had to figure out why it works.
\section{Guts}
Let's represent our number by a sum of digits
\begin{equation}\label{eqn:divisibleByThree:10}
n
= \sum_{k = 0}^N a_k 10^k,
\end{equation}
and the condition for digits adding up to a multiple of three (say \(m\) times \(3\)) is
\begin{equation}\label{eqn:divisibleByThree:30}
\sum_{k = 0}^N a_k = 3 m.
\end{equation}

We can pull that out of the sum, and sure enough, the remainder (in base 10) is divisible by three
\begin{equation}\label{eqn:divisibleByThree:50}
\begin{aligned}
n
&= \sum_{k = 0}^N a_k + \sum_{k = 1}^N a_k (10^k - 1) \\
&= 3m + 9 a_1 + 99 a_2 + 999 a_3 + \cdots \\
&= 3 ( m + 3 a_1 + 33 a_2 + 333 a_3 + \cdots ).
\end{aligned}
\end{equation}

Aurora also mentioned that this works for \(9\) too, which we can see by inspection.  Pretty cool (just like my daughter).
% this is to produce the sites.google url and version info and so forth (for blog posts)
%\vcsinfo
%\EndArticle
%\EndNoBibArticle

%
% Copyright � 2012 Peeter Joot.  All Rights Reserved.
% Licenced as described in the file LICENSE under the root directory of this GIT repository.
%

%
%
%\documentclass{article}

%\input{../peeters_macros.tex}






\chapter{The cross product in three and more dimensions}
\label{chap:cross}
%\author{Peeter Joot \quad peeterjoot@protonmail.com }
\date{ October 12, 2007.  cross.tex }

%\begin{document}

%\maketitle{}

\section{Introduction}

We are initially taught that there are two ways to multiply vectors.  One is the dot product and one is the cross product.  However, we are also taught
how to work with higher dimensional vectors.  There is lots of real applications of higher dimensional vectors that do not require one to waste time puzzling about what the fourth dimension is or how to visualize it.  All we need are four, or N measured qualities for something and we have higher dimensions (position, color, texture, ...).

We see the dot product in a lot of very basic math and physics.  It is inherently simple to understand and to teach, and work with it for a variety of
sorts of calculations.

On the other hand, the cross product is an ugly arbitrary seeming sort of beast.  However, it is a beast that
describes many sorts of physical and mathematical situations.  In vector calculus
cross product terms and it relative the determinant end up occurring all over the place,
and in physics the cross product also occurs in many contexts.
Examples are Stokes theorem, Jacobian transformations, normal equations, the
curl operator, Maxwell's equations, torque, and the list goes on.

The cross product and the dot product have some similarities in form
yet the cross product is only defined for \R{3}, while the dot product
can be defined for \R{n} including \(n > 3\), and even be extended easily in many other ways.  Examples are complex vector spaces, and
the integral dot products that underpin the simple and beautiful Fourier theory.
Note that we call the dot product the inner product when being general, leaving dot or scalar product for the our standard Cartesian vector space beastie.

In many of these cases the mathematics ought to have no logical tie to three dimensions, yet
the cross product is an explicitly three dimensional sort of beast.

It is a bit surprising that one can get all the way through four years of engineering school and still not have an answer about how to do all the sort
of routine vector calculations that we can do in \R{3} in \R{N} (or even \R{4}).
With the second of our two ways to multiply vectors valid only in \R{3}, there is something wrong or
at least missing.

This paper was initially a write up of my scribblings on dot and cross products, where I tried to go back to some of the basic
physical situations that we first the dot and cross products and see for myself how each of these surface in a natural way.  I felt this ought to help indicate where the
explicit three dimensionality of the cross product really came from.

I had some success with this and would say that I had a better feel for the underlying structure of the cross product,
so next came an attempt to investigate possible generalizations of the cross
product to higher dimensions and other mathematical fields.

Since writing some initial notes on this (and producing a generalized cross product that is probably
not entirely useful), I have stumbled across the calculus of differential forms and its wedge product.
I have found, unsurprisingly in retrospect, that I am not the first one to try to
generalize the cross product and the math related to it.
However, differential forms (or calculus on manifolds) can be presented in a horrendously abstract unfriendly fashion, with few of the
geometric considerations that
likely inspired the subject in the first place.
It would be interesting to look at some of the original works by the founders of the subject and see how they presented it.
What we find in texts now is a presentation of the subject in the fashion required to demonstrate its proof from principle to principle.  This hides the natural progression of the subject and is hard to learn from in my opinion.  Obviously there is value to having a thoroughly proven set of
theorems underlying the subject, but that should not inhibit learning or teaching something that appears to have the potential to simplify
the way we work with vector calculus and physics in general.
I have also found a severe lack of clearly worked examples and easy to understand spelled out details in any of my books on this subject, and I have accumulated some of these for myself here.  It may not all make sense without also reading a or some books on differential forms too, but if you are reading this and you are not me, then I hope you get something out of it!

What is this wedge product?  The wedge product is not a mapping into \R{1} or \R{N} from the product of two (or more) vectors in \R{N}, but is instead a value in a ``product space'' that has a dimension usually different from the original.  This is not unreasonable, since we should not have any specific reason to require the product of two vectors itself be a vector of the same dimension.  In fact this is one of the aspects of the cross product that is particularly awkward since we should expect that we can take two vectors in \R{2} and multiply them without having to introduce a normal in a direction that was originally not defined.

We are shown that the wedge product is like the \R{3} cross or triple product in many ways.   It can be used to express the area of the
parallelogram formed by two vectors, or the parallelepiped formed by three, and can also be used to express normals.
My texts on the subject either leave it to you to demonstrate
these for yourself for \R{2} or \R{3} cases, or do not touch on the details at all.  Perhaps it was too ``obvious'' to the author to bother with.

This paper will provide a natural progressive discussion on some of the ways we can end up with a cross product.  An explicit calculation
of the area and volume of an \R{N} parallelogram and parallelepiped will be presented.  Details on how to calculate and express normals to lines,
planes, and volumes will be given in a few different ways.

In the end all this will be tied to the wedge product, and I will hopefully have
an intuitive underpinning that will help me learn differential calculus a bit better.  I had like to see how basic vector math and physics
ends up expressed in the language of differential forms and whether or not it simplifies things and ends up as an easier way to
calculate (I have the feeling it will).

\section{Vector products in geometry}

One can define a product of two vectors in any way you see fit.  The dot product provides a mapping
from \R{n} \(\otimes\) \R{n} to \R{1}, whereas the cross product is a mapping from
\R{3} \(\otimes\) \R{3} to \R{3}.

The introduction of the cross product with a direction that seems so arbitrarily picked in a normal direction to both vectors was one of my reasons for needing to examine the underlying structure.
It is not unreasonable to ask for a vector product definition that
maps a pair of vectors in \R{2} into \R{2}, without introducing a third dimension to express the result
as is required by the cross product.

It seems irregular to have to introduce a product that is in a space different from the original.  In retrospect that is not too irregular, the dot product does just that, as it provides a mapping to a scalar space.

We should expect to be able to define vector products in many different spaces, and the dot and cross products are just two such potential vector products.

Much more general than either of the dot or cross products would be a product that included all the possible pairs of products of the components

\begin{equation}
\Bu \times \Bv = \sum_{i,j}{u_i v_j F_{i,j}}
\end{equation}

Here, \(F_{i,j}\) is some arbitrary function of the indices, perhaps a vector or component of a matrix or a value in some arbitrary field.  One also does not have to assume that there is any relation between \(F_{i,j}\) and \(F_{j,i}\).
This is generally called a tensor product I believe and like the wedge product it is in a space different than the original.  One could for example, express such a product as a NxN matrix or a vector in \R{N^2}.

Let us start with the saner (well, at least more regular and intuitive) older brother of the cross product, the dot product.  Later, examinations of physical concepts, area's, normals, and
volumes will lead to the cross and wedge products.

\subsection{The dot product}

The first time
we see the dot product in physics is in the context of vector projections onto two or three axis.  For example, when drawing
vector force diagrams and calculating work done against a force applied in a direction different than the motion.

Quantifying this projective operation requires nothing more than basic trigonometry, and we can
express the component of a vector
\(\Bv\) in the direction of \(\Bu\) , as:

\begin{equation}\label{eqn:cross:21}
\Proj_{\ucap}(\Bv) = \norm{\Bv} \cos(\theta) \ucap
\end{equation}

Here \(\ucap = \Bu/\norm{\Bu}\) is the unit vector in the direction of \(\Bu\), and \(\theta\) is the angle between the vector \(\Bu\) and \(\Bv\) where the angle is measured with counterclockwise rotation positive as usual.

Unlike the
cross product, both vector projection, and the lengths of the sides of
a triangle defined by vectors in a plane are just as valid in \R{n}
as in \R{3}, and in fact do not even have a particularly strong tie to
the field of real numbers.  There is however a requirement for both concepts
to introduce a distance metric.

The length of a line from the origin \((0, 0)\) to a point \((v_1, v_2)\), can be
shown to be equal to \(\sqrt{v_1^2 + v_2^2}\) (sophisticated math is not required for this and one can show this with the classic square inscribed in a square diagram ... probably dating back to Pythagoras).
Successive applications of this result shows that this length for a point \((v_1, v_2, v_3)\) equals \(\sqrt{v_1^2 + v_2^2 + v_3^2}\).  It is thus natural to define the length of an \R{n} vector in the same fashion: \(\norm{\Bv} = \sqrt{\sum_{i}{v_i^2}}\).

Taking the length of a vector sum (the opposing side of a triangle formed by these two vectors end to end) we have:

\begin{equation}\label{eqn:cross:901}
\begin{aligned}
\norm{\Bu + \Bv}^2 &= \sum_i{(u_i + v_i)(u_i + v_i)} \\
                   &= \sum_i{{u_i}^2 + 2 u_i v_i + {v_i}^2} \\
                   &= \sum_i{{u_i}^2} + 2 \sum_i{u_i v_i} + \sum{{v_i}^2} \\
                   &= \norm{\Bu}^2 + 2 \sum_i{u_i v_i} + \norm{\Bv}^2
\end{aligned}
\end{equation}

So, if these vectors form a right triangle, the middle term \(\sum_i{u_i v_i}\) must equal zero.

Similarly, for a triangle formed by the difference of two vectors both at the origin the length of the opposing side is:

\begin{equation}\label{eqn:cross:921}
\begin{aligned}
\norm{\Bv - \Bu}^2 &= \sum_i{(v_i - u_i)(v_i - u_i)} \\
                   &= \sum_i{{v_i}^2 - 2 u_i v_i + {u_i}^2} \\
                   &= \sum_i{{v_i}^2} - 2 \sum_i{u_i v_i} + \sum{{u_i}^2} \\
                   &= \norm{\Bv}^2 - 2 \sum_i{u_i v_i} + \norm{\Bu}^2
\end{aligned}
\end{equation}

Note that this is just the triangle law.

\begin{equation}\label{eqn:cross:trianglelaw}
\norm{\Bu - \Bv}^2
= \norm{\Bu}^2 + \norm{\Bv}^2 - 2 \norm{\Bu} \norm{\Bv} \cos(\theta)
\end{equation}

This middle term
\(\sum_i{u_i v_i}\)
in both expansions, we give the name (ie: define) ``dot product'', and write that as:

\begin{equation}
\dotprod{\Bu}{\Bv} = \sum_i{u_i v_i}
\end{equation}

By the triangle law comparison above this can also be expressed, writing \(\theta = (\Bu, \Bv)\), in the projective form

\begin{equation}
\dotprod{\Bu}{\Bv} = \norm{\Bu} \norm{\Bv} \cos(\Bu, \Bv)
\end{equation}

Alternatively the projective operation itself can be expressed in terms of the dot product:

\begin{equation}\label{eqn:cross:941}
\begin{aligned}
\Proj_{\ucap}(\Bv) &= (\norm{\Bv} \cos(\theta)) \ucap \\
                  &= \left(\frac{\dotprod{\Bu}{\Bv}}{\norm{\Bu}}\right) \ucap \\
                  &= \bdotprod{\ucap}{\Bv} \ucap
\end{aligned}
\end{equation}

\subsection{Proof of the projective form of the dot product}

Since the proof of the triangle law was not given, our result for the projective form of the dot product is also unproven.
However, a proof of this would also implicitly prove the triangle law
by comparison as above.

Let us do that proof of the projective form of the dot product without requiring a previous (trigonometric) proof of the triangle law.  To calculate \(\cos(\theta)\) where \(\theta\) is the angle between two vectors \(\Bu\) and \(\Bv\), we let

\begin{equation}\label{eqn:cross:961}
\begin{aligned}
\Bx &= \Proj_{\ucap}(\Bv) \\
    &= \alpha\Bu
\end{aligned}
\end{equation}

and,
\begin{equation}\label{eqn:cross:41}
\By = \Bv - \alpha\Bu
\end{equation}

Temporarily imposing a restriction \(\theta \in [0, \pi/2]\) so that \(\alpha\) is positive, we can now express the vector \(\Bv\) in terms of its perpendicular components \(\Bx\) and \(\By\).

\begin{equation}\label{eqn:cross:dotcosine}
\cos(\theta) = \frac{\alpha \norm{\Bu}}{\norm{\Bv}}
\end{equation}

\begin{equation}\label{eqn:cross:981}
\begin{aligned}
\norm{\Bv}^2       &= \\
\norm{\Bx + \By}^2 &= \norm{\alpha\Bu}^2 + \norm{\Bv - \alpha\Bu}^2 \\
                   &= 2 \alpha^2 \norm{\Bu}^2 + \norm{\Bv}^2 - 2\alpha\dotprod{\Bu}{\Bv}
\end{aligned}
\end{equation}

So,
\begin{equation}\label{eqn:cross:1001}
\begin{aligned}
\alpha\dotprod{\Bu}{\Bv} &= \alpha^2 \norm{\Bu}^2 \\
      \dotprod{\Bu}{\Bv} &= \alpha \norm{\Bu}^2 \\
                         &= (\alpha \norm{\Bu}) \norm{\Bu} \\
                         &= \left(\frac{\alpha \norm{\Bu}}{\norm{\Bv}}\right) \norm{\Bu} \norm{\Bv} \\
                         &= \cos(\theta) \norm{\Bu} \norm{\Bv}
\end{aligned}
\end{equation}

Lifting the restriction and considering the \(\theta \in [\pi/2, \pi]\) range, then by the above:

\begin{equation}\label{eqn:cross:1021}
\begin{aligned}
      \dotprod{-\Bu}{\Bv} &= \alpha \norm{\Bu}^2 \\
                          &= (\cos(\pi - \theta)) \norm{-\Bu} \norm{\Bv}
\end{aligned}
\end{equation}

So, again we have:

\begin{equation}\label{eqn:cross:projectivedotprod}
      \dotprod{\Bu}{\Bv}  = \cos(\theta) \norm{\Bu} \norm{\Bv}
\end{equation}

Proof of \eqnref{eqn:cross:projectivedotprod} for the third and fourth quadrants is similar, proving the following:

\begin{equation}\label{eqn:cross:1041}
\begin{aligned}
\cos(\theta)      &= \cos(\Bu,\Bv) \\
                  &= \frac{\dotprod{\Bu}{\Bv}}{\norm{\Bu} \norm{\Bv}} \\
\cos(\ucap,\vcap) &= \dotprod{\ucap}{\vcap}
\end{aligned}
\end{equation}

Now a triangle law, which gave us the significance of the dot product before even naming it, is proven as a side effect.

\subsection{Projective form of the cross product.  Part I.  Normal to vector in direction of other vector}

Since we started with the projective form of the dot product, it is natural to also start with the
projective form of the cross product.

The cross product is first seen (by me at least), in high school was in the following projective form:

\begin{equation}\label{eqn:cross:61}
\crossprod{\Bu}{\Bv} = \norm{\Bu}\norm{\Bv} \sin(\theta) \ncap
\end{equation}

Now, comparing to the projective form of the dot product we can expect that this is going to be
related to the
component of \(\Bv\) that is perpendicular to \(\Bu\), since that vector has magnitude:

\begin{equation}\label{eqn:cross:81}
\norm{\Bv - \Proj_{\ucap}(\Bv)} = \norm{\Bv} | \sin(\theta) |
\end{equation}

Let us calculate the Cartesian representation for the component of \(\Bv\) normal to \(\Bu\), a
perpendicular projection, \(\Proj_{\perp\ucap}(\Bv) = \Bv - \Proj_{\ucap}(\Bv)\):

\begin{equation}\label{eqn:cross:1061}
\begin{aligned}
\Proj_{\perp\ucap}(\Bv) = \Bv - \Proj_{\ucap}(\Bv) &= \Bv - \bdotprod{\ucap}{\Bv} \ucap \\
                                                 &= \frac{1}{\norm{\Bu}^2} \left(\Bv \norm{\Bu}^2 - \bdotprod{\Bu}{\Bv} \Bu \right) \\
                                                 &= \frac{1}{\norm{\Bu}^2} \sum_{i,j}(v_i \ecap_i u_j u_j - u_j v_j u_i \ecap_i) \\
                                                 &= -\frac{1}{\norm{\Bu}^2} \sum_{i,j}u_j \ecap_i (u_i v_j - u_j v_i) \\
                                                 &= -\frac{1}{\norm{\Bu}^2} \sum_{i,j}u_j \ecap_i \DETuvij{u}{v}{i}{j}
\end{aligned}
\end{equation}

For brevity, let us introduce a shorthand notation for this determinant:

\begin{equation}
D_{ij}^{\Bu \Bv} = \DETuvij{u}{v}{i}{j}
\end{equation}

Since \(D_{ii}^{\Bu \Bv} = 0\), we can write:
\begin{equation}\label{eqn:cross:1081}
\begin{aligned}
\Proj_{\perp\ucap}(\Bv) = \Bv - \Proj_{\ucap}(\Bv)
   &= -\frac{1}{\norm{\Bu}^2} \sum_{i,j} u_j \ecap_i D_{ij}^{\Bu \Bv} \\
   &= -\frac{1}{\norm{\Bu}^2} \left(\sum_{i<j} u_j \ecap_i D_{ij}^{\Bu \Bv} + \sum_{j<i} u_{j} \ecap_{i} D_{ij}^{\Bu \Bv}\right) \\
   &= -\frac{1}{\norm{\Bu}^2} \left(\sum_{i<j} u_j \ecap_i D_{ij}^{\Bu \Bv} + \sum_{j'<i'} u_{j'} \ecap_{i'} D_{i'j'}^{\Bu \Bv}\right) \\
   &= -\frac{1}{\norm{\Bu}^2} \left(\sum_{i<j} u_j \ecap_i D_{ij}^{\Bu \Bv} + \sum_{i<j} u_i \ecap_j D_{ji}^{\Bu \Bv}\right) \\
   &= \frac{1}{\norm{\Bu}^2} \sum_{i<j} (u_i \ecap_j - u_j \ecap_i) D_{ij}^{\Bu \Bv}
\end{aligned}
\end{equation}

But \(u_i \ecap_j - u_j \ecap_i\) is also a determinant, so writing \(\Be = (\ecap_1, \dots, \ecap_N)\), we have:

\begin{equation}\label{eqn:cross:1101}
\begin{aligned}
\Proj_{\perp\ucap}(\Bv) = \Bv - \Proj_{\ucap}(\Bv)
   &= \frac{1}{\norm{\Bu}^2} \sum_{i<j} (u_i \ecap_j - u_j \ecap_i) D_{ij}^{\Bu \Bv} \\
   &= \frac{1}{\norm{\Bu}^2} \sum_{i<j} D_{ij}^{\Bu \Bv} D_{ij}^{\Bu \Be}
\end{aligned}
\end{equation}

This has squared magnitude:

\begin{equation}\label{eqn:cross:1121}
\begin{aligned}
\norm{\Proj_{\perp\ucap}(\Bv)}^2
   &= \dotprod{\Bv}{\left(\Bv - \Proj_{\ucap}(\Bv)\right)} \\
   &= \frac{1}{\norm{\Bu}^2} \sum_{i<j} (D_{ij}^{\Bu \Bv})^2
\end{aligned}
\end{equation}

Taking the root:
\begin{equation}\label{eqn:cross:101}
\norm{\Proj_{\perp\ucap}(\Bv)}
   = \frac{1}{\norm{\Bu}} \left(\sum_{i<j} (D_{ij}^{\Bu \Bv})^2\right)^{1/2}
\end{equation}

This also yields the area of the \R{N} parallelogram, with the two vectors \(\Bu\), \(\Bv\) as edges:
\begin{equation}\label{eqn:cross:1141}
\begin{aligned}
\Area(\Bu,\Bv) &= \norm{\Bu} \norm{\Proj_{\perp\ucap}(\Bv)} \\
              &= \left(\sum_{i<j} (D_{ij}^{\Bu \Bv})^2\right)^{1/2}
\end{aligned}
\end{equation}

\subsection{Projective form of the cross product.  Part II.  Normal to two vectors in direction of other vector}

The result \(\Proj_{\perp\ucap}(\Bv) = \frac{1}{\norm{\Bu}^2} \sum_{i<j} D_{ij}^{\Bu \Bv} D_{ij}^{\Bu \Be}\)
does not look like the cross product, and it is not.  However, it is also not a normal to two vectors as the cross product is, only one.

If we continue with the calculation of the normal to two vectors (in the direction of a third) something that we can
calculate in \R{N}, it is expected that the result will have similar aspects to the cross product, especially for \R{3}.

Like the one vector normal
\(\Proj_{\perp\ucap}(\Bv)\)
, we can only calculate this definitively for most dimensions when that calculation is with respect to an additional
vector.  Without a reference vector, we can calculate only specific cases.  These are the \R{2} normal to one vector, and a \R{3} normal to two vectors (cross product), or the \R{N} normal to \(N-1\) vectors (and for all of these the result can vary by an arbitrary scalar multiplier).

It is a bit laborious, but let us calculate the normal to two vectors in the direction of a third (the component that is perpendicular to the plane formed by all the linear combinations of the first two vectors).

Let \(\Bm = \Proj_{\perp\ucap}(\Bv)\)

\begin{equation}\label{eqn:cross:1161}
\begin{aligned}
\Proj_{\perp\ucap,\vcap}(\Bw) &= \Bw - \bdotprod{\Bw}{\ucap}
 \ucap - \bdotprod{\Bw}{\mcap} \mcap \\
                             &= \Bw - \frac{1}{\norm{\Bu}^2}
\bdotprod{\Bu}{\Bw}
 \Bu - \frac{1}{\norm{\Bm}^2}\bdotprod{\Bw}{\Bm} \Bm \\
                             &= \frac{1}{\norm{\Bu}^2 \norm{\Bm}^2}\left(\Bw\norm{\Bu}^2\norm{\Bm}^2 - \norm{\Bm}^2
\bdotprod{\Bu}{\Bw}
 \Bu - {\norm{\Bu}^2}\bdotprod{\Bw}{\Bm} \Bm\right)
\end{aligned}
\end{equation}

Expanding, \(\norm{\Bm}^2\), yields:

\begin{equation}\label{eqn:cross:1181}
\begin{aligned}
\norm{\Bm}^2 &=
\dotprod{\left( \Bv - \bdotprod{\ucap}{\Bv} \ucap \right)} {\Bv} \\
             &= \frac{1}{\norm{\Bu}^2}
\bdotprod{\Bv \norm{\Bu}^2 - \bdotprod{\Bu}{\Bv} \Bu}{\Bv} \\
             &= \frac{1}{\norm{\Bu}^2}
\left( \norm{\Bu}^2 \norm{\Bv}^2 - {\bdotprod{\Bu}{\Bv}}^2 \right) \\
\end{aligned}
\end{equation}
%\norm{\Bm}^2 \norm{\Bu}^2 = \norm{\Bu}^2 \norm{\Bv}^2 - {\bdotprod{\Bu}{\Bv}}^2
%\Bm = \Bv - \bdotprod{\ucap}{\Bv}\ucap

\begin{equation}\label{eqn:cross:1201}
\begin{aligned}
\Rightarrow \Proj_{\perp\ucap,\vcap}(\Bw)
&= \frac{1}{\sum_{i<j} (D_{ij}^{\Bu \Bv})^2}
   \left(\Bw\norm{\Bu}^2\norm{\Bm}^2 - \norm{\Bm}^2
\bdotprod{\Bu}{\Bw}
\Bu - {\norm{\Bu}^2}\bdotprod{\Bw}{\Bm} \Bm\right) \\
\end{aligned}
\end{equation}

\begin{equation}\label{eqn:cross:1221}
\begin{aligned}
\Rightarrow \Proj_{\perp\ucap,\vcap}(\Bw) {\sum_{i<j} (D_{ij}^{\Bu \Bv})^2} &=
\Bw \left(\norm{\Bu}^2 \norm{\Bv}^2 - {\bdotprod{\Bu}{\Bv}}^2\right) \\
&- \left(\frac{1}{\norm{\Bu}^2}\left( \norm{\Bu}^2 \norm{\Bv}^2 -
{\bdotprod{\Bu}{\Bv}}^2\right)\right) {\bdotprod{\Bu}{\Bw}} \Bu \\
&- {\norm{\Bu}^2}
\bdotprod{\Bw}{(\Bv - {\bdotprod{\ucap}{\Bv}}\ucap)}
 \left( \Bv - {\bdotprod{\ucap}{\Bv}}\ucap \right)
\end{aligned}
\end{equation}

\begin{equation}\label{eqn:cross:1241}
\begin{aligned}
\Rightarrow \Proj_{\perp\ucap,\vcap}(\Bw) {\sum_{i<j} (D_{ij}^{\Bu \Bv})^2} \norm{\Bu}^2
&= \Bw \left(\norm{\Bu}^4 \norm{\Bv}^2 - {\bdotprod{\Bu}{\Bv}}^2\norm{\Bu}^2\right) \\
&- \left( \norm{\Bu}^2 \norm{\Bv}^2 - {\bdotprod{\Bu}{\Bv}}
^2\right) {\bdotprod{\Bu}{\Bw}} \Bu \\
&- \left( \dotprod{\Bw}({ {\norm{\Bu}^2} \Bv -
{\bdotprod{\Bu}{\Bv}}
\Bu})\right) \left( \Bv {\norm{\Bu}^2} - {\bdotprod{\Bu}{\Bv}}\Bu\right)
\end{aligned}
\end{equation}

\begin{equation}\label{eqn:cross:1261}
\begin{aligned}
&=
   \left( \norm{\Bu}^4 \norm{\Bv}^2 - {\bdotprod{\Bu}{\Bv}}^2\norm{\Bu}^2 \right)                                             \Bw \\
&- \left( {\bdotprod{\Bv}{\Bw}}
 \norm{\Bu}^4 - {\bdotprod{\Bu}{\Bw}}
{\bdotprod{\Bu}{\Bv}} {\norm{\Bu}^2} \right)                \Bv \\
&+ \left( {\bdotprod{\Bv}{\Bw}}
 {\bdotprod{\Bu}{\Bv}}
 {\norm{\Bu}^2} - {\bdotprod{\Bu}{\Bw}} \norm{\Bu}^2 \norm{\Bv}^2 \right)  \Bu
\end{aligned}
\end{equation}

\begin{equation}\label{eqn:cross:1281}
\begin{aligned}
\Rightarrow \Proj_{\perp\ucap,\vcap}(\Bw) {\sum_{i<j} (D_{ij}^{\Bu \Bv})^2}
&= \left( \norm{\Bu}^2 \norm{\Bv}^2 - {\bdotprod{\Bu}{\Bv}}^2 \right)                              \Bw \\
&- \left( \norm{\Bu}^2 {\bdotprod{\Bv}{\Bw}}
 - {\bdotprod{\Bu}{\Bv}}
 {\bdotprod{\Bu}{\Bw}}\right)    \Bv \\
&+ \left( {\bdotprod{\Bu}{\Bv}}
 {\bdotprod{\Bv}{\Bw}}
 - {\bdotprod{\Bu}{\Bw}} \norm{\Bv}^2 \right)   \Bu \\
\end{aligned}
\end{equation}
\begin{equation}\label{eqn:cross:1301}
\begin{aligned}
= \sum_{ijk} \ecap_i ( &\left( u_j u_j v_k v_k - u_j v_j u_k v_k \right) w_i \\
                      +&\left( u_j w_j u_k v_k - u_j u_j v_k w_k \right) v_i \\
                      +&\left( u_j v_j v_k w_k - u_j w_j v_k v_k \right) u_i )
\end{aligned}
\end{equation}
\begin{equation}\label{eqn:cross:1321}
\begin{aligned}
= \sum_{ijk} u_j v_k \ecap_i ( &\left( u_j v_k - v_j u_k \right) w_i \\
                              +&\left( w_j u_k - u_j w_k \right) v_i \\
                              +&\left( v_j w_k - w_j v_k \right) u_i )
\end{aligned}
\end{equation}
\begin{equation}\label{eqn:cross:1341}
\begin{aligned}
&= \sum_{ijk} u_j v_k \ecap_i \left(
u_i D_{jk}^{\Bv \Bw}
+v_i D_{jk}^{\Bw \Bu}
+w_i D_{jk}^{\Bu \Bv}\right)  \\
&= \sum_{i,j<k} \left(u_j v_k - u_k v_j\right) \ecap_i \left( u_i D_{jk}^{\Bv \Bw} + v_i D_{jk}^{\Bw \Bu} + w_i D_{jk}^{\Bu \Bv} \right) \\
&= \sum_{i,j<k} \ecap_i D_{jk}^{\Bu \Bv} \left( u_i D_{jk}^{\Bv \Bw} + v_i D_{jk}^{\Bw \Bu} + w_i D_{jk}^{\Bu \Bv} \right) \\
&= \sum_{i,j<k} \ecap_i D_{jk}^{\Bu \Bv} D_{ijk}^{\Bu \Bv \Bw} \\
&= \sum_{i<j<k} \ecap_i D_{jk}^{\Bu \Bv} D_{ijk}^{\Bu \Bv \Bw}
+ \sum_{j'<i'<k} \ecap_i' D_{j'k}^{\Bu \Bv} D_{i'j'k}^{\Bu \Bv \Bw}
+ \sum_{j'<k'<i'} \ecap_i' D_{j'k'}^{\Bu \Bv} D_{i'j'k'}^{\Bu \Bv \Bw} \\
&= \sum_{i<j<k} \ecap_i D_{jk}^{\Bu \Bv} D_{ijk}^{\Bu \Bv \Bw}
+ \sum_{i<j<k} \ecap_j D_{ik}^{\Bu \Bv} D_{jik}^{\Bu \Bv \Bw}
+ \sum_{i<j<k} \ecap_k D_{ij}^{\Bu \Bv} D_{kij}^{\Bu \Bv \Bw} \\
\end{aligned}
\end{equation}
\begin{equation}\label{eqn:cross:1361}
\begin{aligned}
&= \sum_{i<j<k} \left(\ecap_i D_{jk}^{\Bu \Bv} + \ecap_j D_{ki}^{\Bu \Bv} + \ecap_k D_{ij}^{\Bu \Bv}\right) D_{ijk}^{\Bu \Bv \Bw} \\
&= \sum_{i<j<k} D_{ijk}^{\Bu \Bv \Bw} D_{ijk}^{\Bu \Bv \Be} \\
\end{aligned}
\end{equation}
\begin{equation}\label{eqn:cross:121}
\Rightarrow \Proj_{\perp\ucap,\vcap}(\Bw) =
\frac{1}{\sum_{i<j} \left(D_{ij}^{\Bu \Bv}\right)^2}\sum_{i<j<k} D_{ijk}^{\Bu \Bv \Bw} D_{ijk}^{\Bu \Bv \Be} \\
\end{equation}

Calculating the magnitude of this vector yields a formula for the volume of the \R{N} parallelepiped spanned by three vectors \(\Bu\), \(\Bv\), and \(\Bw\).

\begin{equation}\label{eqn:cross:1381}
\begin{aligned}
\Volume(\Bu, \Bv, \Bw) &= \Area(\Bu, \Bv) \norm{\Proj_{\perp\ucap, \vcap}(\Bw)} \\
                      &= \left(\sum_{i<j<k} \left(D_{ijk}^{\Bu \Bv \Bw}\right)^{2}\right)^{1/2}
\end{aligned}
\end{equation}

\subsection{Summary of \texorpdfstring{\R{N}}{ND} directed normal results}

\begin{equation}\label{eqn:cross:1401}
\begin{aligned}
\Proj_{\perp\ucap}(\Bv)
   &= \Bv - {\bdotprod{\ucap}{\Bv}}\ucap \\
   &= \Bv - \norm{\Bu}^{-2}{\bdotprod{\Bu}{\Bv}}\Bu \\
   &= \frac{1}{\norm{\Bu}^2} \sum_{i<j} D_{ij}^{\Bu \Bv} D_{ij}^{\Bu \Be} \\
   &= \norm{\Bu}^{-2} \sum_{i<j} \DETuvij{u}{v}{i}{j} \DETuvij{u}{\ecap}{i}{j}
\end{aligned}
\end{equation}
\begin{equation}\label{eqn:cross:1421}
\begin{aligned}
\Proj_{\perp\ucap,\vcap}(\Bw)
&= \Bw - \norm{\Bu}^{-2} {\bdotprod{\Bu}{\Bw}}\Bu \\
&- \norm{\Proj_{\perp\ucap}\left(\Bv\right)}^{-2}
\bdotprod{\Proj_{\perp\ucap}\left(\Bv\right)}{\Bw}
 \Proj_{\perp\ucap}(\Bv) \\
&= \frac{1}{\sum_{i<j} \left(D_{ij}^{\Bu \Bv}\right)^2}\sum_{i<j<k} D_{ijk}^{\Bu \Bv \Bw} D_{ijk}^{\Bu \Bv \Be} \\
&= \left(\sum_{i<j} {\DETuvij{u}{v}{i}{j}}^2\right)^{-1} \sum_{i<j<k} \DETuvwijk{u}{v}{w}{i}{j}{k} \DETuvwijk{u}{v}{\ecap}{i}{j}{k} \\
\end{aligned}
\end{equation}

These results are valid for \R{3}, as well as other dimensions \R{N} (including \R{2} which the cross product is not).

For \R{2} the one vector normal, and for \R{3} the two vector normals become:

\begin{equation}\label{eqn:cross:1441}
\begin{aligned}
\Proj_{\perp\ucap}(\Bv)        &= \left(\norm{\Bu}^{-2} \DETuvij{u}{v}{1}{2}\right) \DETuvij{u}{\ecap}{1}{2} \\
                              &= (scalar value) \DETuvij{u}{\ecap}{1}{2} \\
                              &= (scalar value) \VectorTwo{-u_2}{u_1} \\
\Proj_{\perp\ucap,\vcap}(\Bw)
&= \left(\left(\sum_{1 \leq i<j \leq 3} {\DETuvij{u}{v}{i}{j}}^2\right)^{-1} \DETuvwijk{u}{v}{w}{1}{2}{3}\right) \DETuvwijk{u}{v}{\ecap}{1}{2}{3} \\
&= (scalar value) \DETuvwijk{u}{v}{\ecap}{1}{2}{3} \\
%&= (scalar value) \VectorThree {D_{23}^{\Bu\Bv}} {D_{31}^{\Bu\Bv}} {D_{12}^{\Bu\Bv}} \\
&= (scalar value) \VectorThree {u_2 v_3 - u_3 v_2} {u_3 v_1 - u_1 v_3} {u_1 v_2 - u_2 v_1} \\
%&= (scalar value) (\crossprod{\Bu}{\Bv})
\end{aligned}
\end{equation}

The first is a (scaled) normal vector to \(\Bu\), and the second is a (scaled) normal vector to \(\Bu\) and \(\Bv\) (ie: cross product)

\subsection{Summary of \texorpdfstring{\R{N}}{ND} Area, Volume results}

\begin{equation}\label{eqn:cross:1461}
\begin{aligned}
\Area(\Bu, \Bv)
   &= \left(\sum_{i<j} \left(D_{ij}^{\Bu \Bv}\right)^2\right)^{1/2} \\
   &= \left(\sum_{i<j} {\DETuvij{u}{v}{i}{j}}^2\right)^{1/2} \\
\end{aligned}
\end{equation}
\begin{equation}\label{eqn:cross:1481}
\begin{aligned}
\Volume(\Bu, \Bv, \Bw)
&= \left(\sum_{i<j<k} \left(D_{ijk}^{\Bu \Bv \Bw}\right)^2\right)^{1/2} \\
&= \left(\sum_{i<j<k} {\DETuvwijk{u}{v}{w}{i}{j}{k}}^2\right)^{1/2}
\end{aligned}
\end{equation}

For the \R{2} Area, and \R{3} Volume these becomes the familiar determinant and triplet product results:

\begin{equation}\label{eqn:cross:1501}
\begin{aligned}
\Area(\Bu, \Bv)
&= \abs{ D_{ij}^{uv} } \\
&= abs \DETuvij{u}{v}{1}{2}
\end{aligned}
\end{equation}

\begin{equation}\label{eqn:cross:1521}
\begin{aligned}
\Volume(\Bu, \Bv, \Bw)
&= \abs{\dotprod{(\crossprod{\Bu}{\Bv})}{\Bw}} \\
&= abs \DETuvwijk{u}{v}{w}{1}{2}{3}
\end{aligned}
\end{equation}

What was not proved is that this generalizes, and that the m-parallelepiped volume is what we would expect:

\begin{equation}\label{eqn:cross:141}
\Volume(\Bu_1, \dotsc, \Bu_{m}) =
\sum_{i_1 < \dotsb < i_m} \left(D_{i_1,\dotsc,i_m}^{\Bu_1, \dotsc, \Bu_{m}}\right)^2
\end{equation}

This \(\Volume()\) result corresponds to
the length of a vector, area of a parallelogram, and the volume of a parallelepiped for the 1 vector, 2 vector and 3 vector cases respectively, and this has been proved for \R{N} (not just \R{2} or \R{3}).  To prove it for a
\(m+1\) vector parallelogram, in terms of the
\(\Volume(\Bu_1, \dotsc, \Bu_{m})\)
 we need to take the component of this \(m+1\)'th vector that is perpendicular to the span of all the vectors in the m-parallelepiped and multiply the length of that projection by \(\Volume(\Bu_1, \dotsc, \Bu_{m})\)

Without calculation, it is expected that this perpendicular projection is:

\begin{equation}\label{eqn:cross:1541}
\begin{aligned}
\Proj_{\perp \ucap_1\dotsb\ucap_{m-1}}(\Bu_m)
&=
\Bu_m - \Proj_{\ucap_1\dotsb\ucap_{m-1}}(\Bu_m) \\
&=
\left(\Volume(\Bu_1, \dotsc, \Bu_{m-1})\right)^{-2}
\sum_{i_1 < \dotsb < i_m}
D_{i_1,\dotsc,i_m}^{\Bu_1, \dotsc, \Bu_{m}}
D_{i_1,\dotsc,i_m}^{\Bu_1, \dotsc, \Bu_{m-1}, \Be}
\end{aligned}
\end{equation}

From which the \(m>3\) volume result would follow.  I have not been successful proving this for myself inductively even for
\(m=3\) based on the \(m=2\) result in the form above.  I also note that the books that I have also do not prove this.  Some have a kind of
sneaky way of dealing with this by defining the generalized volume in terms of the wedge product, and never really
demonstrating the geometrical validity of doing so except for \(m=2\) or \(m=3\), or in some cases only for \R{3}.  Since the generalized volume result seems to be uniformly accepted, I am sure some
sufficiently talented mathematician has done the inductive proof for this, perhaps tackling the problem
from some other direction where the result follows more easily.

\section{Normals without a reference vector}

Calculation of a normal above required a reference vector, since there can be normals in many different directions to a set of \(n \in\) \R{N} vectors unless \(n = N-1\).  In the \(n = N-1\) case, the normal only varies by a scalar multiplier.

In this section the normal to a set of vectors will be calculated without introducing a
reference vector.  This is closer to the formulation one would expect of a generalized cross
product, but does generally introduce a set of undetermined coefficients.  We can then
compare the to results for the normals taken in the direction of a reference vector.

\subsection{orthogonality and the Null Space of two vectors}

Using just an
orthogonality condition is not enough to uniquely define a ``cross product''
even in \R{3}, but for \R{3} that is good within at least a scalar multiple.

For \R{N}, lets calculate via row reduction the Null Space of a matrix with rows formed of the elements of the two vectors \(\Bu\) and \(\Bv\), and then solve for \(\Bn\).

\begin{equation}\label{eqn:cross:161}
\begin{bmatrix}
u_1 & u_2 & \dotsb & u_N \\
v_1 & v_2 & \dotsb & v_N
\end{bmatrix}
\VectorN{n}
= \Bzero
\end{equation}

The row reduction can be performed with any set of two columns, not just the first two (the first two could be all zeros for example).  So for generality, we Row reduce based on columns \(i\) and \(j\):

\begin{equation}\label{eqn:cross:181}
\begin{bmatrix}
v_j & -u_j \\
-v_i & u_i
\end{bmatrix}
\begin{bmatrix}
u_1 & u_2 & \dotsb & u_N \\
v_1 & v_2 & \dotsb & v_N
\end{bmatrix}
\VectorN{n} = \Bzero
\end{equation}

For column \(k\) this is:
\begin{equation}\label{eqn:cross:201}
\begin{bmatrix}
v_j & -u_j \\
-v_i & u_i
\end{bmatrix}
\VectorTwo{u_k}{v_k}
=
\VectorTwo{ u_k v_j - u_j v_k }{ u_i v_k - u_k v_i }
=
\VectorTwo{ D_{kj}^{\Bu\Bv} }{ D_{ik}^{\Bu\Bv} }
\end{equation}

In particular, for columns \(i\), and \(j\) respectively this is:

\begin{equation}\label{eqn:cross:221}
\VectorTwo{ D_{ij}^{\Bu\Bv} }{ 0 }
,
\VectorTwo{ 0 }{ D_{ij}^{\Bu\Bv} }
\end{equation}

And we are left with two sets of equations in \(N-2\) dependent variables.

\begin{equation}\label{eqn:cross:1561}
\begin{aligned}
D_{ij}^{\Bu\Bv} n_i &= \sum_{k \neq i,j}D_{jk}^{\Bu\Bv} n_k \\
D_{ij}^{\Bu\Bv} n_j &= \sum_{k \neq i,j}D_{ki}^{\Bu\Bv} n_k \\
\end{aligned}
\end{equation}

Now, let \(n_k = t_k\) for \(k \neq i,j\), and \(t_k\) is an arbitrary constant, and combining these equations:

\begin{equation}\label{eqn:cross:1581}
\begin{aligned}
D_{ij}^{\Bu\Bv} \Bn
&= \sum_s D_{ij}^{\Bu\Bv} n_s \ecap_s \\
&= D_{ij}^{\Bu\Bv} n_i \ecap_i + D_{ij}^{\Bu\Bv} n_j \ecap_j +
\sum_{k \neq i,j} n_k \ecap_k D_{ij}^{\Bu\Bv} \\
&=
\ecap_i \sum_{k \neq i,j}{D_{jk}^{\Bu\Bv} t_k} +
\ecap_j \sum_{k \neq i,j}{D_{ki}^{\Bu\Bv} t_k} +
\sum_{k \neq i,j} t_k \ecap_k D_{ij}^{\Bu\Bv} \\
&=
\sum_{k \neq i,j} t_k
\left(
\ecap_i {D_{jk}^{\Bu\Bv}} +
\ecap_j {D_{ki}^{\Bu\Bv}} +
\ecap_k {D_{ij}^{\Bu\Bv}}
\right) \\
&=
\sum_{k \neq i,j} t_k D_{ijk}^{\Bu\Bv\Be}
\end{aligned}
\end{equation}

Since \(i\) and \(j\) were chosen arbitrarily, this is really the sum over all sets of unique combinations of \(i\), \(j\), and \(k\), so we can write the most
generic normal to a pair of vectors in \R{N} as
\begin{equation}\label{eqn:cross:1601}
\begin{aligned}
\Bn
&= \sum_{i,j,k} s_{ijk} D_{ijk}^{\Bu\Bv\Be} \\
&= \sum_{i<j<k} \left( s_{ijk} - s_{ikj} + s_{kij} - s_{kji} + s_{jki} - s_{jik} \right) D_{ijk}^{\Bu\Bv\Be} \\
&= \sum_{i<j<k} \left( \sum_{\pi_x(i,j,k)} s_{\pi_x}\Sgn(\pi_x) \right) D_{ijk}^{\Bu\Bv\Be} \\
&= \sum_{i<j<k} \left( \sum_{\pi_x \in \pi(i,j,k)} s_{\pi_x}\Sgn(\pi_x) \right) \DETuvwijk{u}{v}{\ecap}{i}{j}{k} \\
\end{aligned}
\end{equation}

Here \(s_{ijk} = t_k/{D_{ij}^{\Bu\Bv}}\), and \(\pi(i,j,k)\) are the permutations of the indices \(i\), \(j\), and \(k\), and \(\Sgn\) is the sign of the individual permutation \(\pi_x\) in that set (-1 for odd numbers of index switches, 1 for even numbers of switches).

For \R{3}, we once again have a scaled cross product.  Also observe that the coefficient term is much like a determinant, the sign alternates with the switch of any two indices and zero if any indices match.  This is not surprising given the earlier calculation of the normal in the direction of a reference vector, was in fact a determinant.

Because the values \(s_{ijk}\) were arbitrary constants, so is the composite value \(s_{ijk}' = \sum_{\pi_x \in \pi(i,j,k)} s_{\pi_x}\Sgn(\pi_x)\), so really this is just a statement that:

\begin{equation}\label{eqn:cross:241}
\Bn \in \Span \left\lbrace
\DETuvwijk{u}{v}{\ecap}{i}{j}{k}
\right\rbrace
\end{equation}

A small note here about the dependence of this result on the field of real numbers.  The normal that was calculated here is not normal for \C{N}, but is a sort of
conjugate normal.  One would have to row reduce the complex conjugates of the vectors to produce a result that is valid for \C{N}.  If one replaces the components in the determinants with their conjugates that should correct the result.

\subsection{orthogonality and the Null Space of one vector}

Having done the calculation for the two vector case, a result like \(\Proj_{\perp\ucap}(\Bv)\) is expected.  Here is the calculation that verifies this:

\begin{equation}\label{eqn:cross:1621}
\begin{aligned}
 (n_i u_i) \ecap_i &= \left( -\sum_{j \neq i}n_j u_j \right) \ecap_i \\
                   &= \left( -\sum_{j \neq i}t_j u_j \right) \ecap_i
\end{aligned}
\end{equation}
\begin{equation}\label{eqn:cross:1641}
\begin{aligned}
 u_i (n_j \ecap_j) &= u_i ( t_j \ecap_j)
\end{aligned}
\end{equation}
\begin{equation}\label{eqn:cross:1661}
\begin{aligned}
 \Rightarrow  u_i \Bn  &= \sum_{j \neq i}( u_i t_j \ecap_j - t_j u_j \ecap_i ) \\
                       &= \sum_{j \neq i} t_j ( u_i \ecap_j - u_j \ecap_i ) \\
                       &= \sum_{j \neq i} t_j D_{ij}^{\Bu\Be}
\end{aligned}
\end{equation}
Let \(s_{ij} = t_j/u_i\),
\begin{equation}\label{eqn:cross:1681}
\begin{aligned}
 \Rightarrow \Bn       &= \sum_{i<j} (s_{ij} - s_{ji}) D_{ij}^{\Bu\Be} \\
                       &= \sum_{i<j} \left(\sum_{\pi_x \in \pi(i,j)}s_{\pi_x} \Sgn(\pi_x)\right) D_{ij}^{\Bu\Be} \\
                       &= \sum_{i<j} s_{ij}' D_{ij}^{\Bu\Be}
\end{aligned}
\end{equation}

With the expected result:

\begin{equation}\label{eqn:cross:261}
\Bn \in \Span \left\lbrace
\DETuvij{u}{\ecap}{i}{j}
\right\rbrace
\end{equation}

\subsection{orthogonality and the Null Space of three vectors}

The calculation of \(\Proj_{\perp\ucap\vcap}(\Bw)\) was pretty laborious.  Without actually calculating it
the expected result for, it is expected that:

\begin{equation}\label{eqn:cross:281}
\Proj_{\perp\ucap\vcap\wcap}(\Bx) = {\Volume(\Bu, \Bv, \Bw)}^{-2} \sum_{i<j<k<l} D_{ijkl}^{\Bu\Bv\Bw\Bx} D_{ijkl}^{\Bu\Bv\Bw\Be}
\end{equation}

There is probably a way to verify this inductively, but it is not obvious to me how to approach this.  One the
other hand the calculation of the Null Space of three vectors is not too hard.

First calculate the Cofactor matrix for
\begin{equation}\label{eqn:cross:301}
\begin{bmatrix}
u_i & u_j & u_k \\
v_i & v_j & v_k \\
w_i & w_j & w_k \\
\end{bmatrix}
\xrightarrow{Transpose}
\begin{bmatrix}
u_i & v_i & w_i \\
u_j & v_j & w_j \\
u_k & v_k & w_k \\
\end{bmatrix}
\xrightarrow{Cofactors}
\begin{bmatrix}
D_{jk}^{\Bv\Bw} & -D_{jk}^{\Bu\Bw} & D_{jk}^{\Bu\Bv} \\
-D_{ik}^{\Bv\Bw} & D_{ik}^{\Bu\Bw} & -D_{ik}^{\Bu\Bv} \\
D_{ij}^{\Bv\Bw} & -D_{ij}^{\Bu\Bw} & D_{ij}^{\Bu\Bv}
\end{bmatrix}
\end{equation}
\begin{equation}\label{eqn:cross:321}
\Rightarrow
\begin{bmatrix}
D_{jk}^{\Bv\Bw} & -D_{jk}^{\Bu\Bw} & D_{jk}^{\Bu\Bv} \\
-D_{ik}^{\Bv\Bw} & D_{ik}^{\Bu\Bw} & -D_{ik}^{\Bu\Bv} \\
D_{ij}^{\Bv\Bw} & -D_{ij}^{\Bu\Bw} & D_{ij}^{\Bu\Bv}
\end{bmatrix}
\VectorThree{u_m}{v_m}{w_m}
=
\begin{bmatrix}
u_m D_{jk}^{\Bv\Bw} & -v_m D_{jk}^{\Bu\Bw} & w_m D_{jk}^{\Bu\Bv} \\
u_m D_{ki}^{\Bv\Bw} & -v_m D_{ki}^{\Bu\Bw} & w_m D_{ki}^{\Bu\Bv} \\
u_m D_{ij}^{\Bv\Bw} & -v_m D_{ij}^{\Bu\Bw} & w_m D_{ij}^{\Bu\Bv}
\end{bmatrix}
=
\VectorThree
{D_{mjk}^{\Bu\Bv\Bw}}
{D_{mki}^{\Bu\Bv\Bw}}
{D_{mij}^{\Bu\Bv\Bw}}
\end{equation}

Columns \(m=i\), \(m=j\), and \(m=k\) are respectively,
\begin{equation}\label{eqn:cross:341}
\VectorThree
{D_{ijk}^{\Bu\Bv\Bw}}
{0}
{0}
,
\VectorThree
{0}
{D_{ijk}^{\Bu\Bv\Bw}}
{0}
,
\VectorThree
{0}
{0}
{D_{ijk}^{\Bu\Bv\Bw}}
\end{equation}

With the introduction of free parameters \(n_m = t_m\) we have four equations,

\begin{equation}\label{eqn:cross:1701}
\begin{aligned}
{D_{ijk}^{\Bu\Bv\Bw}} n_i \ecap_i &= - \ecap_i \sum_{m \neq i,j,k} t_m {D_{mjk}^{\Bu\Bv\Bw}} \\
{D_{ijk}^{\Bu\Bv\Bw}} n_j \ecap_j &= - \ecap_j \sum_{m \neq i,j,k} t_m {D_{mki}^{\Bu\Bv\Bw}} \\
{D_{ijk}^{\Bu\Bv\Bw}} n_k \ecap_k &= - \ecap_k \sum_{m \neq i,j,k} t_m {D_{mij}^{\Bu\Bv\Bw}} \\
\sum_{m \neq i,j,k} {D_{ijk}^{\Bu\Bv\Bw}} n_m \ecap_m &= \sum_{m \neq i,j,k} {D_{ijk}^{\Bu\Bv\Bw}} t_m \ecap_m
\end{aligned}
\end{equation}

Adding these
\begin{equation}\label{eqn:cross:1721}
\begin{aligned}
{D_{ijk}^{\Bu\Bv\Bw}} \Bn
&= \sum_{m} {D_{ijk}^{\Bu\Bv\Bw}} n_m \ecap_m \\
&=
\sum_{m \neq i,j,k} t_m \left(
  \ecap_m {D_{ijk}^{\Bu\Bv\Bw}}
- \ecap_i {D_{jkm}^{\Bu\Bv\Bw}}
+ \ecap_j {D_{kmi}^{\Bu\Bv\Bw}}
- \ecap_k {D_{mij}^{\Bu\Bv\Bw}}
\right) \\
&=
\sum_{m \neq i,j,k} t_m D_{mijk}^{\Bu\Bv\Bw\Be}
\end{aligned}
\end{equation}

A result exactly like the one and two vector cases, with the same conclusion:

\begin{equation}\label{eqn:cross:361}
\Rightarrow
\Bn \in \Span \left\lbrace
\DETuvwxijkl{u}{v}{w}{\ecap}{i}{j}{k}{l}
\right\rbrace
\end{equation}

\subsection{orthogonality and complex vector products}
The same thing can be done for the complex inner product, where
for orthogonality the term,
\begin{equation}\label{eqn:cross:381}
\sum_i{ u_i \overline{v_i} + v_i \overline{u_i}}
\end{equation}
must be zero.

If \(\sum_i{ u_i \overline{v_i}} = 0\), this implies
\(\overline{\sum_i{ u_i \overline{v_i}}} = \sum_i{v_i \overline{u_i}} = 0\), so the definitions of both the
complex and the real inner products arise naturally from an examination of orthogonality constraints.

\section{Introducing the wedge product}

Now, some clever mathematicians have observed that the underlying properties of the determinant is what is important in all these problems of normal, area, and volume.

In particular the property that it changes sign if two of its elements (rows or columns) are switched, and is zero if any of its elements are the same, and it is linear in either variable.

%linearity demo/proof (using cofactor expansion on first row ... any other would do).
% det(a+b, c) = (a_1 + b_1)c_2 - (a_2 + b_2)c_1 = det(a,c) + det(b,c)
% det(a+b, ..., c) = \sum(a_i + b_i)C_1i = \sum(a_i)C_1i +\sum(b_i)C_1i = det(a,...c) + det(b,...,c)

The wedge product is an operator defined based on these two properties.  Symbolically, for two vectors \(\Bu\) and \(\Bv\), these rules are:

\begin{equation}\label{eqn:cross:1741}
\begin{aligned}
\Bu \wedge \Bv &= - (\Bv \wedge \Bu) \\
\Bu \wedge \Bu &= \Bzero \\
(\Bu + \Bv) \wedge \Bw &= \Bu \wedge \Bw + \Bv \wedge \Bw
\end{aligned}
\end{equation}

And for a third vector the wedge defined as:
\begin{equation}\label{eqn:cross:401}
\Bu \wedge \Bv \wedge \Bw = (\Bu \wedge \Bv) \wedge \Bw = \Bu \wedge (\Bv \wedge \Bw)
\end{equation}

Now, intuitively this is a more awkward sort of product than the dot or cross product.  It is it is own thing, in general having no direct mapping to a vector (like the cross product), nor to a scalar (like the dot product).  We will see the dimensions of this beast is not even necessarily close to the dimensions of the original vector space (and that dimension varies according to how many wedge products are composed).

To get a feel for this quantity, an expansion of this in terms of components is helpful (I would have liked to have seen this spelled out in my books for the dumb reader like me).

With
\( \Bu = \sum_i{u_i \ecap_i} \), and \( \Bv = \sum_i{v_i \ecap_i}\), here is the expansion of \(\Bu \wedge \Bv\) .

\begin{equation}\label{eqn:cross:1761}
\begin{aligned}
\Bu \wedge \Bv
&= \left(\sum_i{u_i \ecap_i}\right) \wedge \left(\sum_j{v_j \ecap_j}\right) \\
&= \sum_{i,j}{u_i v_j \left(\ecap_i \wedge \ecap_j\right)} \\
&= \sum_{i \neq j}{u_i v_j \left(\ecap_i \wedge \ecap_j\right)} \\
&= \sum_{i < j}{u_i v_j \left(\ecap_i \wedge \ecap_j\right)} +
   \sum_{j' < i'}{u_{i'} v_{j'} \left(\ecap_{i'} \wedge \ecap_{j'}\right)} \\
&= \sum_{i < j}\left(u_i v_j - u_{j} v_{i}\right)\left(\ecap_i \wedge \ecap_j\right) \\
&= \sum_{i < j}\DETuvij{u}{v}{i}{j}\left(\ecap_i \wedge \ecap_j\right) \\
&= \sum_{i < j}D_{ij}^{\Bu\Bv}\left(\ecap_i \wedge \ecap_j\right)
\end{aligned}
\end{equation}

Introducing a third wedge:
\begin{equation}\label{eqn:cross:1781}
\begin{aligned}
\Bu \wedge \Bv \wedge \Bw
&= \left(\sum_{i < j}D_{ij}^{\Bu\Bv}\left(\ecap_i \wedge \ecap_j\right)\right) \wedge
\Bw = \sum_k{w_k \ecap_k} \\
&= \sum_{i < j, k}w_k D_{ij}^{\Bu\Bv}\left(\ecap_i \wedge \ecap_j \wedge \ecap_k\right) \\
&= \sum_{i < j, k \neq i,j}w_k D_{ij}^{\Bu\Bv}\left(\ecap_i \wedge \ecap_j \wedge \ecap_k\right) \\
&=
\left(\sum_{i < j < k} +
\sum_{i < k < j} +
\sum_{k < i < j}\right)
w_k D_{ij}^{\Bu\Bv}\left(\ecap_i \wedge \ecap_j \wedge \ecap_k\right) \\
&=
\sum_{i < j < k} \left(w_k D_{ij}^{\Bu\Bv}\left(\ecap_i \wedge \ecap_j \wedge \ecap_k\right) +
w_j D_{ik}^{\Bu\Bv}\left(\ecap_i \wedge \ecap_k \wedge \ecap_j\right) +
w_i D_{jk}^{\Bu\Bv}\left(\ecap_j \wedge \ecap_k \wedge \ecap_i\right)\right)
 \\
&=
\sum_{i < j < k} \left(w_k D_{ij}^{\Bu\Bv} - w_j D_{ik}^{\Bu\Bv} + w_i D_{jk}^{\Bu\Bv}\right)
\left(\ecap_i \wedge \ecap_j \wedge \ecap_k\right)
 \\
&=
\sum_{i < j < k} D_{ijk}^{\Bu\Bv\Bw}\left(\ecap_i \wedge \ecap_j \wedge \ecap_k\right)
 \\
\end{aligned}
\end{equation}

Written out in full, these are:
\begin{equation}\label{eqn:cross:1801}
\begin{aligned}
\Bu \wedge \Bv
&=
\sum_{i < j} \DETuvij{u}{v}{i}{j}\left(\ecap_i \wedge \ecap_j\right)  \\
\Bu \wedge \Bv \wedge \Bw
&=
\sum_{i < j < k} \DETuvwijk{u}{v}{w}{i}{j}{k}\left(\ecap_i \wedge \ecap_j \wedge \ecap_k\right)
 \\
\end{aligned}
\end{equation}

A couple observations.

The set
$\left\lbrace \ecap_i \wedge \ecap_j
| i < j
 \right\rbrace
$ is
a basis for the two vector ``wedge product space'', and the set
$\left\lbrace \ecap_i \wedge \ecap_j \wedge \ecap_k
| i < j < k
\right\rbrace
$ is
a basis for the three vector ``wedge product space''.

With these values taken as the basis ``vectors'' for the product space, the natural dot product of two
vectors would be:

\begin{equation}\label{eqn:cross:1821}
\begin{aligned}
\dotprod{(\Bu \wedge \Bv)}{(\Bw \wedge \Bx)}
&=
\dotprod{\left(\sum_{i < j}D_{ij}^{\Bu\Bv}\left(\ecap_i \wedge \ecap_j\right)\right)}
{\left(\sum_{{s} < {t}}D_{{s}{t}}^{\Bw\Bx}\left(\ecap_{s} \wedge \ecap_{t}\right)\right)} \\
&=
\sum_{i < j}\sum_{s < t} D_{ij}^{\Bu\Bv} D_{{s}{t}}^{\Bw\Bx}
\dotprod{\left(\ecap_i \wedge \ecap_j\right)}{\left(\ecap_{s} \wedge \ecap_{t}\right)} \\
&=
\sum_{i < j}\sum_{s < t} D_{ij}^{\Bu\Bv} D_{{s}{t}}^{\Bw\Bx} \delta_{ij,st} \\
&=
\sum_{i < j}D_{ij}^{\Bu\Bv} D_{ij}^{\Bw\Bx}
\end{aligned}
\end{equation}

And in particular, this defines the length of a wedge product, which can be written in terms of the area of the parallelogram spanned by the two ``wedged'' vectors.

\begin{equation}\label{eqn:cross:1841}
\begin{aligned}
\norm{(\Bu \wedge \Bv)}^2
&= \sum_{i < j}\left(D_{ij}^{\Bu\Bv}\right)^2 \\
&= \left(\Area(\Bu,\Bv)\right)^2
\end{aligned}
\end{equation}

For the length of a three vector wedge, we have a length equivalent to the volume of the parallelepiped spanned by the three vectors:
% I think of a proof of this is not neccessary.  Exactly like the above.
\begin{equation}\label{eqn:cross:1861}
\begin{aligned}
\norm{(\Bu \wedge \Bv \wedge \Bw)}^2
&= \sum_{i < j < k}\left(D_{ijk}^{\Bu\Bv\Bw}\right)^2 \\
&= \left(\Volume(\Bu,\Bv,\Bw)\right)^2
\end{aligned}
\end{equation}

So, the elements
\(D_{i_1 \dotsb i_M}^{\Bu^1 \dotsb \Bu^M}\left( \ecap_{i_1} \wedge \dotsb \wedge \ecap_{i_M} \right)\)
of the wedge product
can be thought of as oriented ``\(\Volume\)'' elements of the subspace spanned by the \(M\) vectors.

\subsection{Comparing the wedge product to the normal of \texorpdfstring{\(N-1\)}{N - 1} independent vectors in \texorpdfstring{\R{N}}{R N}}

We have three variations now that generalize the cross product of two vectors in different ways:

\begin{equation}\label{eqn:cross:1881}
\begin{aligned}
\Bu \wedge \Bv &= \sum_{i < j} \DETuvij{u}{v}{i}{j}\left(\ecap_i \wedge \ecap_j\right)  \\
\Proj_{\perp\ucap,\vcap}(\Bw)
&= \left({\sum_{i<j} \left(D_{ij}^{\Bu \Bv}\right)^2}\right)^{-2}\sum_{i<j<k} \DETuvwijk{u}{v}{w}{i}{j}{k} \DETuvwijk{u}{v}{\ecap}{i}{j}{k} \\
\Bn(\Bu,\Bv)
&= \sum_{i<j<k} \left( \sum_{\pi_x \in \pi(i,j,k)} s_{\pi_x}\Sgn(\pi_x) \right) \DETuvwijk{u}{v}{\ecap}{i}{j}{k} \\
\end{aligned}
\end{equation}

The wedge product of \(N-1\) \R{N} vectors and the normal those \(N-1\) vectors (assuming that they are all linearly independent) are a very close match.  Let us compare these for \R{3}, \R{4} and \R{5}.

%\R{2}:
%\begin{align*}
%\Bn(\Bu)
%& \propto \DETuvij{u}{\ecap}{1}{2} \\
%&= (+\ecap_1) |u_2| \\
%&+ (-\ecap_2) |u_1|
%\end{align*}

For \R{3} the normal to two vectors is of the following form:
\begin{equation}\label{eqn:cross:1901}
\begin{aligned}
\Bn(\Bu,\Bv)
&\propto \DETuvwijk{u}{v}{\ecap}{1}{2}{3} \\
&= (+\ecap_1) \DETuvij{u}{v}{2}{3} \\
&+ (-\ecap_2) \DETuvij{u}{v}{1}{3} \\
&+ (+\ecap_3) \DETuvij{u}{v}{1}{2} \\
\end{aligned}
\end{equation}

Compare this to the wedge product of two \R{3} vectors:
\begin{equation}\label{eqn:cross:1921}
\begin{aligned}
\Bu \wedge \Bv
&= (\ecap_2 \wedge \ecap_3) \DETuvij{u}{v}{2}{3} \\
&+ (\ecap_1 \wedge \ecap_3) \DETuvij{u}{v}{1}{3} \\
&+ (\ecap_1 \wedge \ecap_2) \DETuvij{u}{v}{1}{2} \\
\end{aligned}
\end{equation}

Similarly, for \R{4} the normal to three vectors is of the following form:

\R{4}:
\begin{equation}\label{eqn:cross:1941}
\begin{aligned}
\Bn(\Bu,\Bv,\Bw)
&\propto \DETuvwxijkl{u}{v}{w}{\ecap}{1}{2}{3}{4} \\
&= (+\ecap_1) \DETuvwijk{u}{v}{w}{2}{3}{4} + (-\ecap_2) \DETuvwijk{u}{v}{w}{1}{3}{4} \\
&+ (+\ecap_3) \DETuvwijk{u}{v}{w}{1}{2}{4} + (-\ecap_4) \DETuvwijk{u}{v}{w}{1}{2}{3} \\
\end{aligned}
\end{equation}

Compare this to the wedge product of three \R{4} vectors:

\begin{equation}\label{eqn:cross:1961}
\begin{aligned}
\Bu \wedge \Bv \wedge \Bw
&= (\ecap_{2} \wedge \ecap_{3} \wedge \ecap_{4}) \DETuvwijk{u}{v}{w}{2}{3}{4}
+ (\ecap_{1} \wedge \ecap_{3} \wedge \ecap_{4}) \DETuvwijk{u}{v}{w}{1}{3}{4} \\
&+ (\ecap_{1} \wedge \ecap_{2} \wedge \ecap_{4}) \DETuvwijk{u}{v}{w}{1}{2}{4}
+ (\ecap_{1} \wedge \ecap_{2} \wedge \ecap_{3}) \DETuvwijk{u}{v}{w}{1}{2}{3} \\
%\sum_{i < j < k} \DETuvwijk{u}{v}{w}{i}{j}{k}\left(\ecap_i \wedge \ecap_j \wedge \ecap_k\right)
\end{aligned}
\end{equation}

And finally, for \R{5} the normal to four vectors is of the following form:
\begin{equation}\label{eqn:cross:1981}
\begin{aligned}
\Bn(\Bu,\Bv,\Bw,\Bx)
&\propto
%\DETuvwxyijklm{u}{v}{w}{x}{\ecap}{1}{2}{3}{4}{5} \\
\begin{vmatrix}
 {u}_{1} & {u}_{2} & {u}_{3} & {u}_{4} & {u}_{5} \\
 {v}_{1} & {v}_{2} & {v}_{3} & {v}_{4} & {v}_{5} \\
 {w}_{1} & {w}_{2} & {w}_{3} & {w}_{4} & {w}_{5} \\
 {x}_{1} & {x}_{2} & {x}_{3} & {x}_{4} & {x}_{5} \\
 {\ecap}_{1} & {\ecap}_{2} & {\ecap}_{3} & {\ecap}_{4} & {\ecap}_{5}
\end{vmatrix} \\
&= (+\ecap_1) \DETuvwxijkl{u}{v}{w}{x}{2}{3}{4}{5} \\
&+ (-\ecap_2) \DETuvwxijkl{u}{v}{w}{x}{1}{3}{4}{5} \\
&+ (+\ecap_3) \DETuvwxijkl{u}{v}{w}{x}{1}{2}{4}{5} \\
&+ (-\ecap_4) \DETuvwxijkl{u}{v}{w}{x}{1}{2}{3}{4} \\
&+ (+\ecap_5) \DETuvwxijkl{u}{v}{w}{x}{1}{2}{3}{4} \\
\end{aligned}
\end{equation}

Compare this to the wedge product of four \R{5} vectors:
\begin{equation}\label{eqn:cross:2001}
\begin{aligned}
\Bu \wedge \Bv \wedge \Bw \wedge \Bx
%%% \sum_{i < j < k < l} \DETuvwijkl{u}{v}{w}{x}{i}{j}{k}{l}\left(\ecap_i \wedge \ecap_j \wedge \ecap_k \wedge \ecap_l \right)
&= \left(\ecap_{2} \wedge \ecap_{3} \wedge \ecap_{4} \wedge \ecap_{5} \right) \DETuvwxijkl{u}{v}{w}{x}{2}{3}{4}{5} \\
&+ \left(\ecap_{1} \wedge \ecap_{3} \wedge \ecap_{4} \wedge \ecap_{5} \right) \DETuvwxijkl{u}{v}{w}{x}{1}{3}{4}{5} \\
&+ \left(\ecap_{1} \wedge \ecap_{2} \wedge \ecap_{4} \wedge \ecap_{5} \right) \DETuvwxijkl{u}{v}{w}{x}{1}{2}{4}{5} \\
&+ \left(\ecap_{1} \wedge \ecap_{2} \wedge \ecap_{3} \wedge \ecap_{4} \right) \DETuvwxijkl{u}{v}{w}{x}{1}{2}{3}{4} \\
&+ \left(\ecap_{1} \wedge \ecap_{2} \wedge \ecap_{3} \wedge \ecap_{4} \right) \DETuvwxijkl{u}{v}{w}{x}{1}{2}{3}{4} \\
\end{aligned}
\end{equation}

We see that the wedge product is a normal to two vectors in \R{3} (in fact is the cross product) if we make the identities:
\begin{equation}\label{eqn:cross:2021}
\begin{aligned}
 \ecap_1 &= \ecap_2 \wedge \ecap_3 \\
-\ecap_2 &= \ecap_1 \wedge \ecap_3 \\
 \ecap_3 &= \ecap_1 \wedge \ecap_2 \\
\end{aligned}
\end{equation}

And, the wedge product is a normal to three vectors in \R{4} if we make the identities:
\begin{equation}\label{eqn:cross:2041}
\begin{aligned}
  \ecap_1 &= \ecap_{2} \wedge \ecap_{3} \wedge \ecap_{4} \\
 -\ecap_2 &= \ecap_{1} \wedge \ecap_{3} \wedge \ecap_{4} \\
  \ecap_3 &= \ecap_{1} \wedge \ecap_{2} \wedge \ecap_{4} \\
 -\ecap_4 &= \ecap_{1} \wedge \ecap_{2} \wedge \ecap_{3} \\
\end{aligned}
\end{equation}

And, the wedge product is a normal to four vectors in \R{5} if we make the identities:
\begin{equation}\label{eqn:cross:2061}
\begin{aligned}
  \ecap_1 &= \ecap_{2} \wedge \ecap_{3} \wedge \ecap_{4} \wedge \ecap_{5}   \\
 -\ecap_2 &= \ecap_{1} \wedge \ecap_{3} \wedge \ecap_{4} \wedge \ecap_{5}   \\
  \ecap_3 &= \ecap_{1} \wedge \ecap_{2} \wedge \ecap_{4} \wedge \ecap_{5}   \\
 -\ecap_4 &= \ecap_{1} \wedge \ecap_{2} \wedge \ecap_{3} \wedge \ecap_{4}   \\
  \ecap_5 &= \ecap_{1} \wedge \ecap_{2} \wedge \ecap_{3} \wedge \ecap_{4}   \\
\end{aligned}
\end{equation}

So, as well as \(\Bu \wedge \Bv\) being an oriented parallelogram area vector, and \(\Bu \wedge \Bv \wedge \Bw\) being an oriented parallelepiped volume vector,
the wedge product of \(N-1\) linearly independent vectors in \R{N}
is also normal to those vectors if we make the identity:

\begin{equation}\label{eqn:cross:421}
  \ecap_j = \Sgn(j, i_1, \dotsc, i_{N-1}) \ecap_{i_1} \wedge \dotsb \wedge \ecap_{i_{N-1}}
\end{equation}
Or,
\begin{equation}\label{eqn:cross:2081}
\begin{aligned}
\ecap_{i_1} \wedge \dotsb \wedge \ecap_{i_{N-1}} &= \Sgn(j, i_1, \dotsc, i_{N-1}) \ecap_j \\
                                                 &= (-1)^{j+1} \ecap_j
\end{aligned}
\end{equation}

Where \(i_1 < i_2 < \dotsb < i_{N-1}\) and \(j \notin \lbrace i_1, \dotsc, i_{N-1} \rbrace\).

\subsection{Comparing the wedge product and the cross product and triple product}

For the wedge of \(N-1\) vectors in \R{N}, and using the mapping from this wedge product to a normal to these vectors we have:

\begin{equation}\label{eqn:cross:2101}
\begin{aligned}
\Bu^{1} \wedge \dotsb \wedge \Bu^{N-1} &=
\sum_{i_1<\dotsb <i_{N-1}}{ D_{i_1 \dotsb i_{N-1}}^{\Bu^{1}\dotsc\Bu^{N-1}}\left( \ecap_{i_1} \wedge \dotsb \wedge \ecap_{i_{N-1}} \right) } \\
&=
\sum_{i_1<\dotsb <i_{N-1}}{ D_{i_1 \dotsb i_{N-1}}^{\Bu^{1}\dotsc\Bu^{N-1}}
\left(
%\ecap_{i_1} \wedge \dotsb \wedge \ecap_{i_{N-1}}
\Sgn(j, i_1, \dotsc, i_{N-1}) \ecap_j
\right)
} \\
\end{aligned}
\end{equation}

\begin{equation}\label{eqn:cross:Nminus1Product}
\Rightarrow
\Bu^{1} \wedge \dotsb \wedge \Bu^{N-1} =
\sum_{i_1<\dotsb <i_{N-1}}{ D_{i_1 \dotsb i_{N-1}}^{\Bu^{1}\dotsc\Bu^{N-1}}
\left(
%\Sgn(j, i_1, \dotsc, i_{N-1}) \ecap_j
(-1)^{j+1} \ecap_j
\right)
}
\end{equation}

Where, as before \(j \notin \lbrace i_1, \dotsc, i_{N-1} \rbrace\).

(Note that for \R{3} this is exactly the cross product \(\crossprod{\Bu^1}{\Bu^2}\).)

Continuing with the wedge of \(N-1\) vectors as a normal in \R{N} we can write:

\begin{equation}\label{eqn:cross:2121}
\begin{aligned}
\dotprod{\left(\Bu^{1} \wedge \dotsb \wedge \Bu^{N-1}\right)}{\Bu^N}
&=
\sum_{j,i_1<\dotsb <i_{N-1}}
   {
      D_{i_1 \dotsb i_{N-1}}^{\Bu^{1}\dotsc\Bu^{N-1}}
      \left( (-1)^{j+1} u_j^N \right)
   } \\
&=
D_{1 \dotsb N}^{\Bu^{1}\dotsc\Bu^{N}}
\end{aligned}
\end{equation}

But,
\begin{equation}\label{eqn:cross:441}
\Bu^{1} \wedge \dotsb \wedge \Bu^{N} =
D_{1 \dotsb N}^{\Bu^{1}\dotsc\Bu^{N}}
\left(\ecap_{1} \wedge \dotsb \wedge \ecap_{N}\right)
\end{equation}
\begin{equation}\label{eqn:cross:461}
\Rightarrow
\Bu^{1} \wedge \dotsb \wedge \Bu^{N} =
\left(\dotprod{\left(\Bu^{1} \wedge \dotsb \wedge \Bu^{N-1}\right)}{\Bu^N} \right)
\left(\ecap_{1} \wedge \dotsb \wedge \ecap_{N}\right)
\end{equation}

This result is very much like the \R{3} triple product:

\begin{equation}\label{eqn:cross:481}
\tripleprod{\Bu}{\Bv}{\Bw} = \DETuvwijk{u}{v}{w}{1}{2}{3}
\end{equation}

This should not be surprising since \eqnref{eqn:cross:Nminus1Product} was the \R{3} cross product.



Explicitly expanding this \R{N} wedge of \(N\) vectors for a couple of dimensions to illustrate.

For \R{2}:
\begin{equation}\label{eqn:cross:501}
\Bu \wedge \Bv = \DETuvij{u}{v}{1}{2} (\ecap_1 \wedge \ecap_2)
\end{equation}
(Here the determinant is the oriented area of the \R{2} parallelogram formed by vectors \(\Bu\) and \(\Bv\).)

For \R{3}:
\begin{equation}\label{eqn:cross:521}
\Bu \wedge \Bv \wedge \Bw = \DETuvwijk{u}{v}{w}{1}{2}{3} (\ecap_1 \wedge \ecap_2 \wedge \ecap_3)
\end{equation}
(Here the determinant is the oriented volume of the \R{3} parallelepiped formed by vectors \(\Bu\), \(\Bv\), and \(\Bw\).)

And for \R{4}:
\begin{equation}\label{eqn:cross:541}
\Bu \wedge \Bv \wedge \Bw \wedge \Bx = \DETuvwxijkl{u}{v}{w}{x}{1}{2}{3}{4} (\ecap_1 \wedge \ecap_2 \wedge \ecap_3 \wedge \ecap_4)
\end{equation}
(Here the determinant is the oriented \(4\Volume\) of the \R{4} parallelo-4gram formed by vectors \(\Bu\), \(\Bv\), \(\Bw\), and \(\Bx\))

Each of these are one dimensional wedge product space vectors can be thought of as directed
areas, volumes, or N-volumes.  There are two ways that the sign can vary, one is due to the ordering of the vectors themselves, and the
other is due to the ordering of the \(\ecap_{i_{1}} \wedge \dotsb \wedge \ecap_{i_{N}}\) term. Picking the ordering of that
term is equivalent to picking a basis for the wedge product space.  Once that basis is picked it defines an isomorphism with \R{1}.


What this does do is put the wedge product into a context that we are used to.

We can identify the \(N-1\) vector wedge product with the cross product at least with respect to its normal properties.

We can identify the wedge product of \(N\) vectors with the triple product (which is better described as an \(N\) product, valid for \(N \geq 2\)).
This \(N\) product is a scalar, and in particular has
a geometric meaning which is the area, volume, ... of the parallelo-Ngram formed by the span of the vectors in question.

\subsection{Comparing the wedge product and the general normal to some vectors}

In general can we put the normal equations in a form closer to that of the wedge product?

\begin{equation}\label{eqn:cross:2141}
\begin{aligned}
\Proj_{\perp\ucap,\vcap}(\Bw)
\left({\sum_{i<j} \left(D_{ij}^{\Bu \Bv}\right)^2}\right)^2
&=
\sum_{i<j<k} \DETuvwijk{u}{v}{w}{i}{j}{k} \DETuvwijk{u}{v}{\ecap}{i}{j}{k} \\
&=
\sum_{i<j<k} \DETuvwijk{u}{v}{w}{i}{j}{k}
\left(\ecap_i \DETuvij{u}{v}{j}{k}
+\ecap_j \DETuvij{u}{v}{k}{i}
+\ecap_k \DETuvij{u}{v}{i}{j} \right) \\
&=
\sum_{t<i<j} \DETuvwijk{u}{v}{w}{t}{i}{j}
\ecap_t \DETuvij{u}{v}{i}{j} \\
&-
\sum_{i<t<j} \DETuvwijk{u}{v}{w}{i}{t}{j}
\ecap_t \DETuvij{u}{v}{i}{j} \\
&+
\sum_{i<j<t} \DETuvwijk{u}{v}{w}{i}{j}{t}
\ecap_t \DETuvij{u}{v}{i}{j} \\
&=
\sum_{i<j} \left( \sum_t \DETuvwijk{u}{v}{w}{t}{i}{j} \ecap_t\right) \DETuvij{u}{v}{i}{j} \\
\end{aligned}
\end{equation}

The same can be done for the general normal to two vectors,

\begin{equation}\label{eqn:cross:2161}
\begin{aligned}
\Bn(\Bu,\Bv)
&= \sum_{i<j<k} \left( \sum_{\pi_x \in \pi(i,j,k)} s_{\pi_x}\Sgn(\pi_x) \right)
\left(\ecap_i \DETuvij{u}{v}{j}{k}
+\ecap_j \DETuvij{u}{v}{k}{i}
+\ecap_k \DETuvij{u}{v}{i}{j} \right) \\
&=
\sum_{t<i<j}
\left( \sum_{\pi_x \in \pi(t,i,j)} s_{\pi_x}\Sgn(\pi_x) \right)
\ecap_t \DETuvij{u}{v}{i}{j} \\
&- \sum_{i<t<j}
\left( \sum_{\pi_x \in \pi(i,t,j)} s_{\pi_x}\Sgn(\pi_x) \right)
\ecap_t \DETuvij{u}{v}{i}{j} \\
&+\sum_{i<j<t}
\left( \sum_{\pi_x \in \pi(i,j,t)} s_{\pi_x}\Sgn(\pi_x) \right)
\ecap_t \DETuvij{u}{v}{i}{j} \\
&=
\sum_{i<j}
\left( \sum_{t=1}^{n} \ecap_t \sum_{\pi_x \in \pi(t,i,j)} s_{\pi_x}\Sgn(\pi_x) \right)
\DETuvij{u}{v}{i}{j} \\
\end{aligned}
\end{equation}

Let us expand the \(\Proj_{\perp\ucap,\vcap}(\Bw)\) result for a few specific examples \R{3}, \R{4} and \R{5} to get a feel for it.

Here is the \R{3} case:
\begin{equation}\label{eqn:cross:2181}
\begin{aligned}
\Bn_{(\Bu,\Bv)}(\Bw)
&=
\Proj_{\perp\ucap,\vcap}(\Bw)
\left({\sum_{ij=12,23,13} \left(D_{ij}^{\Bu \Bv}\right)^2}\right)^2 \\
&=
\sum_{ij=12,23,13}
\left(\sum_{t=1}^{3} \ecap_t \DETuvwijk{u}{v}{w}{t}{i}{j} \right)\DETuvij{u}{v}{i}{j} \\
&=
\sum_{ij=12,23,13}
\left(\sum_{t=1}^{3} \ecap_t D_{{t}{i}{j}}^{\Bu\Bv\Bw} \right)
\DETuvij{u}{v}{i}{j} \\
&=
\sum_{t=1}^{3} \ecap_t D_{{t}{1}{2}}^{\Bu\Bv\Bw} \DETuvij{u}{v}{1}{2}
+ \sum_{t=1}^{3} \ecap_t D_{{t}{2}{3}}^{\Bu\Bv\Bw} \DETuvij{u}{v}{2}{3}
+ \sum_{t=1}^{3} \ecap_t D_{{t}{1}{3}}^{\Bu\Bv\Bw} \DETuvij{u}{v}{1}{3} \\
&=
 \ecap_3 D_{{3}{1}{2}}^{\Bu\Bv\Bw} \DETuvij{u}{v}{1}{2}
+ \ecap_1 D_{{1}{2}{3}}^{\Bu\Bv\Bw} \DETuvij{u}{v}{2}{3}
+ \ecap_2 D_{{2}{1}{3}}^{\Bu\Bv\Bw} \DETuvij{u}{v}{1}{3} \\
&=
D_{{1}{2}{3}}^{\Bu\Bv\Bw} \left(
 \ecap_3 \DETuvij{u}{v}{1}{2}
+\ecap_1 \DETuvij{u}{v}{2}{3}
+\ecap_2 \DETuvij{u}{v}{3}{1}
\right) \\
&=
\DETuvwijk{u}{v}{w}{1}{2}{3} \left(
 \ecap_3 \DETuvij{u}{v}{1}{2}
+\ecap_1 \DETuvij{u}{v}{2}{3}
+\ecap_2 \DETuvij{u}{v}{3}{1}
\right)
\end{aligned}
\end{equation}

And here is the \R{4} case:
\begin{equation}\label{eqn:cross:2201}
\begin{aligned}
\Bn_{(\Bu,\Bv)}(\Bw)
&=
\Proj_{\perp\ucap,\vcap}(\Bw)
\left({\sum_{ij=12,13,14,23,24,34} \left(D_{ij}^{\Bu \Bv}\right)^2}\right)^2 \\
&=
\sum_{ij=12,13,14,23,24,34} \left( \sum_t \DETuvwijk{u}{v}{w}{t}{i}{j} \ecap_t\right) \DETuvij{u}{v}{i}{j} \\
&=
\sum_{t=1}^{4} \ecap_t D_{{t}{1}{2}}^{\Bu\Bv\Bw} \DETuvij{u}{v}{1}{2}
+
\sum_{t=1}^{4} \ecap_t D_{{t}{1}{3}}^{\Bu\Bv\Bw} \DETuvij{u}{v}{1}{3} \\
&+
\sum_{t=1}^{4} \ecap_t D_{{t}{1}{4}}^{\Bu\Bv\Bw} \DETuvij{u}{v}{1}{4}
+
\sum_{t=1}^{4} \ecap_t D_{{t}{2}{3}}^{\Bu\Bv\Bw} \DETuvij{u}{v}{2}{3} \\
&+
\sum_{t=1}^{4} \ecap_t D_{{t}{2}{4}}^{\Bu\Bv\Bw} \DETuvij{u}{v}{2}{4}
+
\sum_{t=1}^{4} \ecap_t D_{{t}{3}{4}}^{\Bu\Bv\Bw} \DETuvij{u}{v}{3}{4} \\
&=
\left(
\ecap_{3} D_{{3}{1}{2}}^{\Bu\Bv\Bw}
+\ecap_{4} D_{{4}{1}{2}}^{\Bu\Bv\Bw}
\right)
\DETuvij{u}{v}{1}{2}
+
\left(
\ecap_{2} D_{{2}{1}{3}}^{\Bu\Bv\Bw}
+\ecap_{4} D_{{4}{1}{3}}^{\Bu\Bv\Bw}
\right)
\DETuvij{u}{v}{1}{3} \\
&+
\left(
\ecap_{2} D_{{2}{1}{4}}^{\Bu\Bv\Bw}
+\ecap_{3} D_{{3}{1}{4}}^{\Bu\Bv\Bw}
\right)
\DETuvij{u}{v}{1}{4}
+
\left(
\ecap_{1} D_{{1}{2}{3}}^{\Bu\Bv\Bw}
+\ecap_{4} D_{{4}{2}{3}}^{\Bu\Bv\Bw}
\right)
\DETuvij{u}{v}{2}{3} \\
&+
\left(
\ecap_{1} D_{{1}{2}{4}}^{\Bu\Bv\Bw}
+\ecap_{3} D_{{3}{2}{4}}^{\Bu\Bv\Bw}
\right)
\DETuvij{u}{v}{2}{4}
+
\left(
\ecap_{1} D_{{1}{3}{4}}^{\Bu\Bv\Bw}
+\ecap_{2} D_{{2}{3}{4}}^{\Bu\Bv\Bw}
\right)
\DETuvij{u}{v}{3}{4} \\
\end{aligned}
\end{equation}

For \R{5} the set of indices \(\lbrace{ij}\rbrace = \lbrace{12,13,14,15,23,24,25,34,35,45}\rbrace\).  The
coefficients of the \(\DETuvij{u}{v}{i}{j}\) terms are \(\sum_{t \neq i,j} \ecap_t D_{tij}^{\Bu\Bv\Bw}\), so for
\R{5} this will be three terms.

So, we can equate the vector normal to the plane of \(\Bu\) and \(\Bv\) in the direction of \(\Bw\) if one introduces the following mapping:

\begin{equation}\label{eqn:cross:561}
\ecap_i \wedge \ecap_j = \frac{\sum_{t \neq i,j} \ecap_t D_{tij}^{\Bu\Bv\Bw}}
{\sum_{i<j} \left(D_{ij}^{\Bu\Bv}\right)^2}
\end{equation}

Comparing to the normal to the \(\Bu,\Bv\) plane in an unspecified direction:
\(\Bn = \sum s_{ijk}D_{ijk}^{\Bu\Bv\Be}\)
we can treat \(\Bu \wedge \Bv\) as a normal if we write:

\begin{equation}\label{eqn:cross:581}
\ecap_i \wedge \ecap_j =
\left( \sum_{t \neq i \neq j} \ecap_t \sum_{\pi_x \in \pi(t,i,j)} s_{\pi_x}\Sgn(\pi_x) \right)
\end{equation}

Similarly we can think of
\(\Bu \wedge \Bv \wedge \Bw\) as a normal to the \(\Bu,\Bv,\Bw\) volume, \(\Bn = \sum{s_{ijkl}D_{ijkl}^{\Bu\Bv\Bw\Be}}\),  if we write:

\begin{equation}\label{eqn:cross:601}
\ecap_i \wedge \ecap_j \wedge \ecap_k =
\left( \sum_{t \neq i \neq j \neq k} \ecap_t \sum_{\pi_x \in \pi(t,i,j,k)} s_{\pi_x}\Sgn(\pi_x) \right)
\end{equation}

%  Example for \R{4}
%
%\begin{align*}
%\Bn(\Bu,\Bv,\Bw) =
%\left( \sum_{t \neq i \neq j \neq k} \sum_{\pi_x \in \pi(t,i,j,k)} s_{\pi_x}\Sgn(\pi_x) \right)
%\left( \sum_{i<j<k<l} \ecap_
%\sum_{\pi_x \in \pi(t,i,j,k)} s_{\pi_x}\Sgn(\pi_x) \right)
%\end{align*}

This is not exactly a natural seeming correspondence, unlike the \(N-1\) vector case.  There it did make some sense to treat
the wedge product as a normal form, but only in the \(N-1\) vector case is there an unambiguous normal form (apart from the scalar multiplier).

\subsection{A first attempt to tie in to differential forms.  Dot product of wedge products as normal times magnitude}

There is a natural scenario where it does make sense to equate the normal and the wedge.

By example, if one has a surface function with components f, and g in the \(x_1,x_2\) and \(x_2,x_3\) planes respectively, one could in \R{3} express this as:

\begin{equation}\label{eqn:cross:621}
\Bv(\Br) = f(\Br) \ecap_3 + g(\Br) \ecap_1
\end{equation}

As illustrated by the generic normal calculations, in \R{4},
the normal to the \(x_1,x_2\) plane does not have a unique direction and potentially has components in one or more of the \(\ecap_3\) and \(\ecap_4\) directions, and
the normal to the \(x_2,x_3\) plane potentially has \(\ecap_1\) and \(\ecap_4\) components.

So, it is much more natural to not try to express a directed surface function in terms of a normal that is not well defined unless the ``surface'' is of dimension \(N-1\).  Instead such a directed surface function is best expressed exclusively in terms of its components in each of its planes.  Restating the above in such terms we have:

\begin{equation}\label{eqn:cross:641}
\Bv(\Br) = f(\Br) \ecap_{1,2} + g(\Br) \ecap_{2,3}
\end{equation}

Here \(\ecap_{1,2}\) indicates that \(f(\Br)\) is the component of the surface function \(\Bv\) that is in the \(x_1,x_2\) plane and \(\ecap_{2,3}\) means that \(g(\Br)\) is the component of \(\Bv\) in the \(x_2,x_3\) plane.

It is natural to use the wedge product to express each of these components, writing:

\begin{equation}\label{eqn:cross:661}
\Bv(\Br) = f(\Br) (\ecap_1 \wedge \ecap_2) + g(\Br) (\ecap_2 \wedge \ecap_3)
\end{equation}

A formulation like this is well regardless of whether the space is \R{3}, \R{4}, \R{5}, or \R{N}, though it does not have any specific physical interpretation.

More generally, a surface function in \R{N} can be expressed as
\begin{equation}\label{eqn:cross:681}
\Bv(\Br) = \sum_{i \neq j} f_{ij}(\Br) \ecap_{i,j}
\end{equation}

and write this as:

\begin{equation}\label{eqn:cross:701}
\Bv(\Br) = \sum_{i<j} f_{ij}(\Br) \ecap_i \wedge \ecap_j
\end{equation}

This and the fact that we can express length, area, volume, n-volumes, as wedge products thus gives us a way to describe flux and work like quantities.

Recall that a line integral of the following form describes phenomena such as ``work done''.

\begin{equation}\label{eqn:cross:702}
\int_{\Br} \dotprod{\Bv(\Br)}{d\Br}
\end{equation}

This is the component of a directed field and multiply by the length of a vector in the specified direction.

Similarly a surface integral of the following form describes ``flux through a surface'' like quantities.  We write this as:

\begin{equation}\label{eqn:cross:721}
\int_{\BA} \dotprod{\Bv(\Br)}{d\BA}
\end{equation}

This is the component of a field in the direction of a surface, is multiplied by the area element for that surface.

Using the example above with components f, and g in the \(x_1,x_2\) and \(x_2,x_3\) planes respectively, one could in \R{3} express the flux for this function as:

\begin{equation}\label{eqn:cross:741}
\int_{\Br} \dotprod{\left(f(\Br) \ecap_3 + g(\Br) \ecap_1\right)}{\left(dx_1 dx_2 \ecap_3 + dx_2 dx_3 \ecap_1 + dx_3 dx_1 \ecap_2\right)}
\end{equation}

But again, this is a formulation that is only good for \R{3}.  Also note that there is an implied common orientation
of the normal and the area elements.

Because we can express the component of a surface function as a wedge product and can also express an oriented area element as a wedge product we can
express this flux quantity in terms of the wedge product and have a formulation that is valid for \R{N} as well as \R{3}.

\begin{equation}\label{eqn:cross:761}
\int
\dotprod{\left(f(\Br) (\ecap_1 \wedge \ecap_3) + g(\Br) (\ecap_2 \wedge \ecap_3)\right)}{\left(\sum_{i<j} d\ecap_i \wedge d\ecap_j\right)}
=
\int f(\Br) dx_1 dx_3 + g(\Br) dx_2 dx_3
\end{equation}

Similarly, say one has a volume function in \R{4} with components f, and g with \(x_1,x_2,x_3\) and \(x_1,x_2,x_4\) cubes, one could express it as:

\begin{equation}\label{eqn:cross:781}
\Bv(\Br) = f(\Br) \ecap_4 + g(\Br) \ecap_3
\end{equation}

But this is only a good description for \R{4}.  If one expressed the same thing as:

\begin{equation}\label{eqn:cross:801}
\Bv(\Br) = f(\Br) (\ecap_1 \wedge \ecap_2 \wedge \ecap_3) + g(\Br) (\ecap_1 \wedge \ecap_2 \wedge \ecap_4)
\end{equation}

This is a good description of the directed volume function for \R{4} as well as any other dimension \R{N}, and one could write a \R{N} volume ``flux'' like integral
for this function as:

\begin{equation}\label{eqn:cross:821}
\int
\dotprod{\left(
f(\Br) (\ecap_1 \wedge \ecap_2 \wedge \ecap_3) + g(\Br) (\ecap_1 \wedge \ecap_2 \wedge \ecap_4)
\right)}
{\left(\sum_{i<j<k} d\ecap_i \wedge d\ecap_j \wedge d\ecap_k\right)}
\end{equation}
\begin{equation}\label{eqn:cross:841}
=
\int f(\Br) dx_1 dx_2 dx_3 + g(\Br) dx_1 dx_2 dx_4
\end{equation}

Here like the surface flux formulation, the volume function has a specific orientation.  That orientation has been defined by
considering its \(i,j,k\) element as positive in the ``direction'' of \(\ecap_i \wedge \ecap_j \wedge \ecap_k\).

Now, it may not look like much has been gained by introducing the wedge product, but when it comes to parameterizing the surface-integral, or volume-integral
this allows for a great deal of flexibility.  For example, if the piece of the surface element can be parametrized as a parallelogram in terms of two vectors \(d\Bu\) and \(d\Bv\), then that element of the flux integral above can be expressed as

\begin{equation}\label{eqn:cross:861}
\int
\dotprod{\left(f(\Br) (\ecap_1 \wedge \ecap_3) + g(\Br) (\ecap_2 \wedge \ecap_3)\right)}{\left(d\Bu \wedge d\Bv\right)}
\end{equation}

and using the volume flux example above, if a parallelepiped volume element is parametrized in terms of three vectors
\(d\Bu\), \(d\Bv\) and \(d\Bw\), then the flux element can be expressed as

\begin{equation}\label{eqn:cross:881}
\int
\dotprod{\left(
f(\Br) (\ecap_1 \wedge \ecap_2 \wedge \ecap_3) + g(\Br) (\ecap_1 \wedge \ecap_2 \wedge \ecap_4)
\right)}{\left(d\Bu \wedge d\Bv \wedge d\Bw\right)}
\end{equation}

%\end{document}               % End of document.
